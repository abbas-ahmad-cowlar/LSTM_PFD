% ============================================================================
% Deep Learning Approaches for Bearing Fault Diagnosis
% A Comprehensive Study of CNN, LSTM, and Hybrid Architectures
% ============================================================================

\documentclass[12pt,a4paper,twoside]{article}

% ============================================================================
% PACKAGES
% ============================================================================

% Typography and Layout
\usepackage[utf8]{inputenc}
\usepackage[T1]{fontenc}
\usepackage{lmodern}
\usepackage{microtype}
\usepackage[margin=1in]{geometry}
\usepackage{setspace}
\usepackage{parskip}

% Mathematics
\usepackage{amsmath}
\usepackage{amssymb}
\usepackage{amsthm}
\usepackage{mathtools}

% Graphics and Figures
\usepackage{graphicx}
\usepackage{subcaption}
\usepackage{float}
\usepackage[dvipsnames,table]{xcolor}
\usepackage{tikz}
\usetikzlibrary{shapes,arrows,positioning,calc}

% Tables
\usepackage{booktabs}
\usepackage{multirow}
\usepackage{makecell}
\usepackage{longtable}
\usepackage{tabularx}

% References and Citations
\usepackage[numbers,sort&compress]{natbib}
\usepackage[hidelinks]{hyperref}
\usepackage{cleveref}

% Algorithms
\usepackage{algorithm}
\usepackage{algpseudocode}

% Code Listings
\usepackage{listings}
\lstset{
    basicstyle=\ttfamily\small,
    breaklines=true,
    frame=single,
    numbers=left,
    numberstyle=\tiny\color{gray},
    keywordstyle=\color{blue},
    commentstyle=\color{ForestGreen},
    stringstyle=\color{red},
    showstringspaces=false,
    tabsize=2
}

% Headers and Footers
\usepackage{fancyhdr}
\pagestyle{fancy}
\fancyhf{}
\fancyhead[LE,RO]{\thepage}
\fancyhead[RE]{\leftmark}
\fancyhead[LO]{\rightmark}
\renewcommand{\headrulewidth}{0.4pt}

% Sectioning
\usepackage{titlesec}
\titleformat{\section}
  {\normalfont\Large\bfseries\color{NavyBlue}}{\thesection}{1em}{}
\titleformat{\subsection}
  {\normalfont\large\bfseries\color{NavyBlue}}{\thesubsection}{1em}{}
\titleformat{\subsubsection}
  {\normalfont\normalsize\bfseries\color{NavyBlue}}{\thesubsubsection}{1em}{}

% Custom Commands
\newcommand{\todo}[1]{\textcolor{red}{\textbf{[TBD: #1]}}}
\newcommand{\figplaceholder}[1]{\textcolor{blue}{\textbf{[Figure: #1]}}}
\newcommand{\tabplaceholder}[1]{\textcolor{blue}{\textbf{[Table: #1]}}}

% Theorems and Definitions
\theoremstyle{definition}
\newtheorem{definition}{Definition}[section]
\newtheorem{example}{Example}[section]

% ============================================================================
% DOCUMENT INFORMATION
% ============================================================================

\title{%
    \vspace{-2cm}
    \Huge\textbf{Deep Learning Approaches for\\[0.3cm]
    Bearing Fault Diagnosis}\\[0.5cm]
    \Large A Comprehensive Study of CNN, LSTM, and Hybrid Architectures\\[1cm]
}

\author{%
    \large Bearing Fault Diagnosis Research Team\\[0.3cm]
    \normalsize Department of Mechanical Engineering\\
    \normalsize Advanced Diagnostics Laboratory\\[0.5cm]
}

\date{\today}

% ============================================================================
% DOCUMENT
% ============================================================================

\begin{document}

% Title Page
\maketitle
\thispagestyle{empty}

\begin{abstract}
Bearing failures are among the leading causes of unplanned downtime in rotating machinery, making early and accurate fault detection critical for industrial operations. This study presents a comprehensive investigation of deep learning approaches for automated bearing fault diagnosis using vibration signal analysis. We systematically evaluate three distinct methodologies: Convolutional Neural Networks (CNNs) for spatial pattern recognition, Long Short-Term Memory (LSTM) networks for temporal sequence modeling, and a novel configurable hybrid architecture that combines both approaches.

Using the Case Western Reserve University bearing dataset comprising 1,430 vibration signals across 11 fault categories, we implement and compare multiple architectural variants. Our CNN-based approach (Milestone 1) employs 15+ architectures including ResNet and EfficientNet families, achieving \todo{XX.XX\%} classification accuracy through effective spatial feature extraction. The LSTM-based approach (Milestone 2) utilizes both unidirectional and bidirectional recurrent networks, demonstrating \todo{XX.XX\%} accuracy by capturing temporal dependencies in sequential data. Most notably, our hybrid CNN-LSTM framework (Milestone 3) introduces a configurable architecture allowing arbitrary combinations of CNN backbones with LSTM types, achieving \todo{XX.XX\%} accuracy while providing flexibility for application-specific optimization.

Experimental results reveal that \todo{[comparative analysis of the three approaches will be inserted here after training completion]}. The hybrid architecture demonstrates particular effectiveness in \todo{[specific scenarios/fault types]}. We provide detailed performance metrics including per-class precision, recall, F1-scores, and confusion matrices for all approaches. Computational efficiency analysis shows \todo{[inference time comparisons]} with \todo{[model size comparisons]}.

This work contributes to the field of intelligent fault diagnosis by: (1) providing a systematic comparison of CNN, LSTM, and hybrid approaches on a standardized bearing dataset, (2) introducing a flexible hybrid architecture framework enabling rapid experimentation with different backbone combinations, (3) delivering production-ready implementations with comprehensive documentation, and (4) establishing performance benchmarks for future research. Our findings have direct implications for industrial predictive maintenance systems, enabling more reliable condition monitoring and reducing operational costs through early fault detection.

\end{abstract}

\vfill

\noindent\textbf{Keywords:} Bearing fault diagnosis, Deep learning, Convolutional Neural Networks, Long Short-Term Memory, Hybrid architectures, Vibration analysis, Predictive maintenance, Condition monitoring

\clearpage

% Table of Contents
\tableofcontents
\clearpage

% List of Figures
\listoffigures
\clearpage

% List of Tables
\listoftables
\clearpage

% ============================================================================
% MAIN CONTENT
% ============================================================================

\onehalfspacing

\section{Introduction}
\label{sec:introduction}
Rotating machinery forms the backbone of modern industrial infrastructure, with applications spanning manufacturing, power generation, transportation, and aerospace systems. Among the critical components in these systems, rolling element bearings play an indispensable role in ensuring smooth operation and load transfer. However, bearing failures account for approximately 40-50\% of all breakdowns in rotating machinery\cite{randall2011rolling}, leading to substantial economic losses through unplanned downtime, production disruptions, and potential catastrophic failures. The ability to detect bearing faults at their nascent stages has therefore become paramount for maintaining operational efficiency and preventing costly equipment failures.

\subsection{Motivation and Industrial Context}

Traditional condition monitoring approaches rely heavily on manual inspection, periodic measurements, and rule-based alarm systems. While these methods have served industry for decades, they suffer from several fundamental limitations. Manual inspection is labor-intensive, subjective, and often fails to detect incipient faults before they escalate. Rule-based systems, though automated, require extensive domain expertise to design effective thresholds and often struggle with the complexity and variability of real-world operating conditions. The increasing complexity of modern machinery, coupled with demands for higher operational efficiency and reduced maintenance costs, necessitates more sophisticated diagnostic approaches.

Vibration-based condition monitoring has emerged as the preferred technique for bearing health assessment due to its non-invasive nature and rich information content. When a bearing develops a fault, it generates characteristic vibration patterns that reflect the underlying damage mechanism. Different fault types---such as outer race defects, inner race defects, ball defects, or cage faults---produce distinct vibration signatures. However, extracting meaningful diagnostic information from raw vibration signals presents significant challenges. The signals are often contaminated with noise, influenced by varying operating conditions, and contain complex interactions between multiple fault components. Moreover, early-stage faults produce subtle changes that are difficult to detect with conventional signal processing techniques.

\subsection{The Deep Learning Revolution in Fault Diagnosis}

The advent of deep learning has fundamentally transformed the landscape of intelligent fault diagnosis. Unlike traditional machine learning approaches that require manual feature engineering, deep learning models can automatically discover hierarchical feature representations directly from raw data. This capability is particularly valuable in fault diagnosis, where the optimal features may be non-obvious and application-specific. Deep neural networks have demonstrated remarkable success in learning discriminative patterns from complex, high-dimensional data, making them well-suited for bearing fault classification tasks.

Convolutional Neural Networks (CNNs), originally developed for computer vision, have shown exceptional performance in processing grid-like data structures. When applied to vibration signals, CNNs can extract local patterns, detect fault-specific impulses, and learn translation-invariant features---properties that align well with the characteristics of bearing vibration data. The hierarchical nature of CNNs allows them to capture features at multiple scales, from fine-grained local patterns to broader structural characteristics.

Recurrent Neural Networks (RNNs), particularly Long Short-Term Memory (LSTM) networks, offer complementary strengths by explicitly modeling temporal dependencies in sequential data. Bearing vibration signals are inherently time-series data, where the temporal ordering and long-range correlations contain critical diagnostic information. LSTMs can capture these temporal dynamics, learning how vibration patterns evolve over time and identifying fault-specific sequential signatures.

The natural question that arises is: can we combine the spatial pattern recognition capabilities of CNNs with the temporal modeling strengths of LSTMs to achieve superior performance? This question motivates our investigation of hybrid architectures that integrate both approaches.

\subsection{Research Objectives}

This study addresses the following research objectives:

\begin{enumerate}
    \item \textbf{Systematic Evaluation of CNN Architectures:} Implement and compare multiple CNN architectures (including ResNet and EfficientNet families) for bearing fault classification, identifying the most effective architectural choices for this application domain.

    \item \textbf{Investigation of LSTM-Based Temporal Modeling:} Develop LSTM-based models that exploit the sequential nature of vibration data, comparing unidirectional and bidirectional variants to assess the importance of temporal context.

    \item \textbf{Development of Configurable Hybrid Architecture:} Design a flexible framework that allows arbitrary combinations of CNN backbones with LSTM types, enabling systematic exploration of hybrid approaches and application-specific optimization.

    \item \textbf{Comprehensive Performance Benchmarking:} Establish rigorous performance metrics and fair comparison protocols to evaluate all approaches on a standardized dataset, providing actionable insights for practitioners.

    \item \textbf{Production-Ready Implementation:} Deliver complete, well-documented implementations that can be directly deployed in industrial settings, bridging the gap between research and practical application.
\end{enumerate}

\subsection{Contributions}

This work makes the following key contributions to the field of intelligent fault diagnosis:

\begin{itemize}
    \item \textbf{Comprehensive Comparative Study:} We provide the first systematic comparison of CNN, LSTM, and hybrid approaches on the same bearing dataset with identical experimental protocols, enabling direct performance comparisons.

    \item \textbf{Novel Configurable Hybrid Framework:} Our hybrid architecture introduces a modular design that decouples CNN feature extraction from LSTM temporal modeling, allowing 42+ different configurations through simple parameter changes. This flexibility enables rapid experimentation and optimization for specific requirements.

    \item \textbf{Multiple Architecture Variants:} We implement and evaluate 15+ CNN architectures, 2 LSTM types, and 3 recommended hybrid configurations, providing extensive empirical evidence about architectural choices.

    \item \textbf{Detailed Performance Analysis:} Beyond overall accuracy metrics, we provide comprehensive analysis including per-class performance, confusion patterns, computational requirements, and failure mode analysis.

    \item \textbf{Open Research Platform:} All implementations are thoroughly documented with training scripts, evaluation tools, and usage examples, facilitating reproducibility and enabling future research.
\end{itemize}

\subsection{Incremental Development Approach}

Our research follows a systematic three-milestone development approach:

\textbf{Milestone 1 (CNN-Based Approach)} focuses on spatial pattern recognition using convolutional architectures. We implement multiple CNN variants ranging from basic architectures to advanced models like ResNet and EfficientNet, establishing baseline performance for pure spatial feature extraction.

\textbf{Milestone 2 (LSTM-Based Approach)} investigates temporal sequence modeling using recurrent architectures. By processing vibration signals as time series, we explore how temporal dependencies and sequential patterns contribute to fault classification performance.

\textbf{Milestone 3 (Hybrid CNN-LSTM Approach)} integrates spatial and temporal modeling through a configurable hybrid framework. This milestone represents the culmination of our research, combining the strengths of both approaches while maintaining architectural flexibility.

\subsection{Organization of This Report}

The remainder of this report is organized as follows. Section~\ref{sec:literature} reviews related work in bearing fault diagnosis, deep learning architectures, and hybrid approaches. Section~\ref{sec:dataset} describes the Case Western Reserve University bearing dataset used in all experiments. Section~\ref{sec:methodology} presents our three methodologies in detail, covering CNN-based, LSTM-based, and hybrid approaches. Section~\ref{sec:experimental} outlines the experimental setup, training procedures, and evaluation metrics. Section~\ref{sec:results} presents comprehensive results and discussion, comparing all approaches across multiple dimensions. Finally, Section~\ref{sec:conclusion} concludes with key findings and directions for future work.


\section{Literature Review}
\label{sec:literature}
This section reviews the evolution of bearing fault diagnosis techniques, with particular emphasis on deep learning approaches that form the foundation of our work.

\subsection{Traditional Fault Diagnosis Methods}

Early bearing fault diagnosis relied primarily on time-domain statistical features extracted from vibration signals, including root mean square (RMS), kurtosis, crest factor, and peak values\cite{tandon1999review}. While computationally efficient, these simple statistical measures often lack the discriminative power needed for accurate fault classification, particularly in noisy environments or with multiple simultaneous faults.

Frequency-domain analysis through Fast Fourier Transform (FFT) became widely adopted due to its ability to reveal characteristic fault frequencies. Each bearing fault type generates specific frequency components related to shaft speed, bearing geometry, and fault location\cite{randall2011rolling}. However, FFT-based approaches assume signal stationarity and struggle with time-varying operating conditions common in industrial settings.

Time-frequency analysis methods, including Short-Time Fourier Transform (STFT), Wavelet Transform, and Hilbert-Huang Transform, address the limitations of pure frequency analysis by providing joint time-frequency representations\cite{yan2014wavelets}. These techniques can capture transient phenomena and non-stationary characteristics. However, they introduce additional complexity in parameter selection (e.g., wavelet type, window size) and still require manual feature engineering for classification.

Envelope analysis and demodulation techniques specifically target the modulation patterns characteristic of bearing faults\cite{antoni2007spectral}. By isolating resonance frequencies and demodulating the signal, these methods can enhance fault signatures. Despite their effectiveness, they require careful selection of frequency bands and filtering parameters, often necessitating expert knowledge.

\subsection{Machine Learning for Fault Diagnosis}

The application of classical machine learning to bearing fault diagnosis began with methods like Support Vector Machines (SVMs), k-Nearest Neighbors (k-NN), and Random Forests\cite{widodo2007support}. These approaches showed improved classification performance compared to simple threshold-based methods, but remained dependent on handcrafted features extracted through signal processing techniques. The quality of feature engineering directly determines classification performance, making these methods labor-intensive and application-specific.

Researchers explored various feature extraction strategies, including statistical features, frequency-domain features, time-frequency features, and entropy-based measures\cite{lei2013artificial}. Feature selection techniques such as Principal Component Analysis (PCA) and Linear Discriminant Analysis (LDA) were employed to reduce dimensionality and improve generalization. While these methods achieved reasonable performance on controlled datasets, they struggled to generalize across different operating conditions and machinery types due to their reliance on manual feature design.

\subsection{Deep Learning Revolution in Fault Diagnosis}

The emergence of deep learning marked a paradigm shift from feature engineering to feature learning. Deep neural networks can automatically discover hierarchical representations directly from raw data, eliminating the need for manual feature design\cite{lecun2015deep}.

\subsubsection{CNN-Based Approaches}

Convolutional Neural Networks have demonstrated exceptional performance in bearing fault diagnosis. Zhang et al.\cite{zhang2017new} pioneered the application of CNNs to bearing fault diagnosis by treating 1D vibration signals as sequential data and applying 1D convolutions. Their work showed that CNNs could automatically learn fault-specific patterns without manual feature extraction. Subsequent research explored various CNN architectures and input representations.

Ince et al.\cite{ince2016real} proposed real-time motor fault detection using 1D CNNs, demonstrating the computational efficiency of convolutional approaches. Janssens et al.\cite{janssens2016convolutional} investigated CNN feature learning for bearing fault diagnosis, analyzing what features are learned at different network depths. They found that shallow layers capture local patterns and impulses, while deeper layers learn more abstract, fault-specific representations.

Several studies converted 1D vibration signals into 2D representations (spectrograms, scalograms, or time-frequency images) to leverage 2D CNN architectures originally designed for computer vision\cite{wen2018new}. While this approach enables the use of pretrained models through transfer learning, it introduces additional preprocessing steps and may not fully exploit the temporal structure of vibration data.

Advanced CNN architectures including ResNet\cite{he2016deep} and DenseNet have been adapted for fault diagnosis, showing improved performance through skip connections and feature reuse\cite{zhang2019deep}. These architectural innovations address the vanishing gradient problem in deep networks and enable training of much deeper models. However, most existing work focuses on specific architecture choices without systematic comparison across multiple CNN families.

\subsubsection{RNN and LSTM-Based Approaches}

Recurrent Neural Networks offer a natural framework for processing sequential vibration data. Zhao et al.\cite{zhao2017deep} applied deep RNNs to bearing fault diagnosis, demonstrating that temporal modeling can capture fault-specific sequential patterns. However, standard RNNs suffer from vanishing gradient problems and struggle with long-range dependencies.

Long Short-Term Memory networks address these limitations through gating mechanisms that regulate information flow\cite{hochreiter1997long}. Yuan et al.\cite{yuan2016fault} applied LSTMs to rolling bearing fault diagnosis, showing superior performance compared to standard RNNs. The ability of LSTMs to remember long-term dependencies proves particularly valuable for capturing subtle fault signatures that evolve over extended time periods.

Bidirectional LSTMs (BiLSTMs) process sequences in both forward and backward directions, providing richer temporal context\cite{graves2005framewise}. Chen et al.\cite{chen2020intelligent} demonstrated that BiLSTMs outperform unidirectional LSTMs in bearing fault diagnosis by capturing both past and future context. However, BiLSTMs double the computational cost and may not be suitable for real-time applications requiring causal processing.

Several studies explored stacked LSTM architectures and attention mechanisms to enhance temporal modeling\cite{zhao2019deep}. While these approaches show promise, they significantly increase model complexity and training time. The trade-off between performance gain and computational cost remains an important consideration for practical deployment.

\subsubsection{Hybrid Approaches}

The complementary strengths of CNNs (spatial pattern recognition) and LSTMs (temporal modeling) motivate hybrid architectures. Pioneering work by Sainath et al.\cite{sainath2015convolutional} in speech recognition demonstrated that CNN-LSTM hybrids could outperform either architecture alone.

In fault diagnosis, several researchers have explored hybrid approaches. Zhao et al.\cite{zhao2016bearing} proposed a CNN-LSTM model for bearing fault diagnosis, using CNNs to extract local features and LSTMs to model temporal dependencies. Their results showed improved performance over pure CNN or LSTM approaches. However, their architecture used a fixed CNN backbone with a specific LSTM configuration, limiting flexibility for optimization.

Wang et al.\cite{wang2019bearing} investigated attention mechanisms in CNN-LSTM models, allowing the network to focus on fault-relevant temporal regions. While attention improves interpretability and can enhance performance, it introduces additional hyperparameters and computational overhead.

Recent work has explored more sophisticated integration strategies, including multi-scale feature extraction, residual connections between CNN and LSTM layers, and ensemble approaches\cite{li2020intelligent}. However, most existing hybrid architectures are tightly coupled, making it difficult to systematically evaluate different CNN-LSTM combinations or optimize for specific requirements.

\subsection{Research Gaps and Opportunities}

Despite significant progress, several research gaps remain:

\begin{enumerate}
    \item \textbf{Lack of Systematic Comparison:} Most studies evaluate specific architectures in isolation, making it difficult to assess relative strengths. Fair comparisons require identical datasets, preprocessing, and evaluation protocols.

    \item \textbf{Limited Architectural Flexibility:} Existing hybrid approaches typically fix both CNN and LSTM components, preventing exploration of alternative combinations that might be better suited for specific applications.

    \item \textbf{Insufficient Analysis of Trade-offs:} Beyond accuracy metrics, practical deployment requires understanding computational costs, memory requirements, inference time, and robustness to noise---dimensions often overlooked in academic studies.

    \item \textbf{Dataset and Operating Condition Variability:} Many studies use limited datasets or controlled laboratory conditions, raising questions about generalization to diverse industrial environments.

    \item \textbf{Reproducibility Challenges:} Lack of standardized implementations and incomplete methodological details hinder reproducibility and make it difficult to build upon prior work.
\end{enumerate}

Our work addresses these gaps through systematic evaluation of CNN, LSTM, and hybrid approaches on standardized datasets, introduction of a configurable hybrid framework enabling exploration of multiple architecture combinations, comprehensive performance analysis across accuracy, efficiency, and robustness dimensions, and complete open-source implementations with detailed documentation. These contributions advance both the scientific understanding of deep learning for fault diagnosis and practical deployment in industrial settings.


\section{Dataset Description}
\label{sec:dataset}
This study utilizes the Case Western Reserve University (CWRU) Bearing Data Center dataset, one of the most widely cited benchmark datasets in bearing fault diagnosis research. This section provides comprehensive details about the dataset characteristics, fault types, data collection methodology, and preprocessing procedures.

\subsection{Experimental Test Rig}

The CWRU dataset was collected from a test rig consisting of a 2-horsepower (hp) motor driving a shaft through a torque transducer/encoder to a dynamometer. Vibration data were measured using accelerometers attached to both the motor housing and the drive end bearing housing. The test rig allowed precise control of motor speed and load conditions, enabling systematic data collection across various operating scenarios.

Accelerometers were positioned at the 12 o'clock position on both the drive end and fan end of the motor housing. The sensors used were industrial-grade accelerometers with high sensitivity and wide frequency response, ensuring accurate capture of bearing vibration characteristics. Data acquisition was performed using a 16-channel DAT recorder, providing high-fidelity signal recording.

\subsection{Bearing Specifications}

The test bearings used in the experiments are SKF deep-groove ball bearings. Table~\ref{tab:bearing_specs} presents the key specifications of the test bearings used in data collection.

\begin{table}[h]
\centering
\caption{Bearing Specifications}
\label{tab:bearing_specs}
\begin{tabular}{@{}ll@{}}
\toprule
\textbf{Parameter} & \textbf{Value} \\
\midrule
Bearing Type & SKF 6205-2RS JEM \\
Number of Rolling Elements & 9 \\
Ball Diameter & 7.94 mm \\
Pitch Diameter & 39.04 mm \\
Contact Angle & 0° \\
\bottomrule
\end{tabular}
\end{table}

These specifications enable calculation of theoretical bearing fault frequencies, which are critical for validating diagnostic results and understanding the physical mechanisms underlying observed vibration patterns.

\subsection{Fault Introduction Methodology}

Faults were introduced into the test bearings using electro-discharge machining (EDM). This controlled fault seeding method ensures reproducible defects with precise dimensions. Three types of single-point defects were created:

\begin{itemize}
    \item \textbf{Inner Race Faults:} Defects machined on the inner race surface
    \item \textbf{Outer Race Faults:} Defects positioned at the 6 o'clock load zone on the outer race
    \item \textbf{Ball Faults:} Defects introduced on the rolling elements
\end{itemize}

For this study, we utilize an extended version of the CWRU dataset that includes additional fault categories beyond the standard inner race, outer race, and ball defects. Our dataset encompasses 11 distinct fault classes representing realistic industrial bearing failure modes.

\subsection{Fault Categories}

Table~\ref{tab:fault_categories} summarizes the 11 fault categories included in our dataset.

\begin{table}[h]
\centering
\caption{Bearing Fault Categories}
\label{tab:fault_categories}
\begin{tabular}{@{}clp{7cm}@{}}
\toprule
\textbf{Class} & \textbf{Fault Type} & \textbf{Description} \\
\midrule
0 & Healthy & Normal bearing operation without defects \\
1 & Misalignment & Shaft or bearing misalignment causing non-uniform load distribution \\
2 & Imbalance & Rotor imbalance leading to synchronous vibration \\
3 & Bearing Clearance & Excessive internal clearance due to wear or manufacturing tolerance \\
4 & Lubrication Issue & Insufficient or contaminated lubrication \\
5 & Cavitation & Fluid cavitation causing surface pitting \\
6 & Wear & Progressive surface degradation through abrasion \\
7 & Oil Whirl & Instability phenomenon in fluid film bearings \\
8 & Mixed Fault 1 & Combination of misalignment and imbalance \\
9 & Mixed Fault 2 & Combination of bearing clearance and lubrication issues \\
10 & Mixed Fault 3 & Combination of wear and cavitation \\
\bottomrule
\end{tabular}
\end{table}

The inclusion of mixed fault categories reflects realistic industrial conditions where multiple degradation mechanisms often occur simultaneously. This multi-class classification problem presents significant challenges for diagnostic algorithms.

\subsection{Data Acquisition Parameters}

Vibration signals were sampled at multiple rates depending on the motor speed:

\begin{itemize}
    \item 12 kHz sampling rate for motor speeds of 1797 rpm
    \item 48 kHz sampling rate for motor speeds of 1772 rpm and 1750 rpm
\end{itemize}

For consistency across all experiments in this study, we standardize the data processing to a uniform sampling rate of \textbf{20.48 kHz}. This rate provides adequate frequency resolution to capture bearing fault signatures while maintaining computational efficiency.

\subsection{Dataset Composition}

Our complete dataset comprises:

\begin{itemize}
    \item \textbf{Total Samples:} 1,430 vibration signal segments
    \item \textbf{Signal Length:} 102,400 samples per segment
    \item \textbf{Duration:} Approximately 5 seconds per segment at 20.48 kHz sampling rate
    \item \textbf{Number of Classes:} 11 fault categories (including healthy condition)
    \item \textbf{Class Distribution:} Approximately 130 samples per fault class
\end{itemize}

The dataset exhibits balanced class distribution, which is advantageous for training deep learning models as it prevents class imbalance issues that could bias the classifier toward majority classes.

\subsection{Data Format}

All vibration data are stored in MATLAB .mat file format, with each file containing:

\begin{itemize}
    \item Raw vibration signal (1D array of acceleration values)
    \item Sampling rate information
    \item Fault type label
    \item Operating condition metadata (motor speed, load)
\end{itemize}

This structured format facilitates reproducible data loading and preprocessing across all experimental implementations.

\subsection{Data Preprocessing}

We apply minimal preprocessing to preserve the raw signal characteristics and allow deep learning models to learn appropriate representations directly from data. The preprocessing pipeline consists of:

\begin{enumerate}
    \item \textbf{Normalization:} Each signal is normalized to zero mean and unit variance:
    \begin{equation}
    x_{norm} = \frac{x - \mu}{\sigma}
    \end{equation}
    where $\mu$ and $\sigma$ are the signal mean and standard deviation, respectively.

    \item \textbf{Segmentation:} Long continuous recordings are segmented into fixed-length windows of 102,400 samples. This length captures multiple rotation cycles while maintaining manageable computational requirements.

    \item \textbf{Quality Control:} Signals are inspected for anomalies such as clipping, excessive noise, or sensor failures. Any corrupted segments are excluded from the dataset.
\end{enumerate}

No additional filtering, feature extraction, or signal transformation is applied, ensuring that deep learning models receive raw vibration data and must learn all relevant features autonomously.

\subsection{Data Splitting Strategy}

To ensure unbiased evaluation and fair comparison across all approaches, we adopt a stratified random split:

\begin{itemize}
    \item \textbf{Training Set:} 60\% of data (approximately 858 samples)
    \item \textbf{Validation Set:} 20\% of data (approximately 286 samples)
    \item \textbf{Test Set:} 20\% of data (approximately 286 samples)
\end{itemize}

Stratification ensures that all fault classes are proportionally represented in each subset, preventing evaluation bias. The same data split is used consistently across all three milestones (CNN, LSTM, and Hybrid approaches) to enable direct performance comparison.

The validation set is used for hyperparameter tuning and model selection during development, while the test set remains held-out for final performance evaluation. This protocol prevents overfitting to the test set and provides realistic estimates of generalization performance.

\subsection{Data Augmentation}

To improve model generalization and robustness, we employ several data augmentation techniques during training:

\begin{enumerate}
    \item \textbf{Additive Gaussian Noise:} Random noise is added to simulate sensor noise and environmental interference:
    \begin{equation}
    x_{aug} = x + \mathcal{N}(0, \alpha^2)
    \end{equation}
    where $\alpha$ controls the noise level (typically 0.01 to 0.05).

    \item \textbf{Amplitude Scaling:} Signals are randomly scaled to simulate varying sensor gains and signal strengths:
    \begin{equation}
    x_{aug} = \beta \cdot x
    \end{equation}
    where $\beta \sim \mathcal{U}(0.8, 1.2)$ is uniformly sampled.

    \item \textbf{Time Shifting:} Circular shifting simulates different starting positions within the rotation cycle:
    \begin{equation}
    x_{aug} = \text{circshift}(x, k)
    \end{equation}
    where $k$ is randomly sampled from $[0, 1024]$.
\end{enumerate}

These augmentation strategies are applied probabilistically during training (with 50\% probability per augmentation) to increase dataset diversity without fundamentally altering fault signatures. Augmentation is not applied to validation or test sets to ensure fair evaluation.

\subsection{Dataset Characteristics and Challenges}

While the CWRU dataset is widely used and well-understood, it presents several challenges:

\begin{itemize}
    \item \textbf{Controlled Environment:} Data collected under laboratory conditions may not fully represent the complexity of industrial environments with multiple noise sources and varying operating conditions.

    \item \textbf{Single Bearing Type:} All data come from the same bearing model, which may limit generalization to other bearing geometries and manufacturers.

    \item \textbf{Synthetic Faults:} EDM-introduced faults, while reproducible, may differ from naturally occurring degradation in real-world applications.

    \item \textbf{Limited Mixed Faults:} While we include three mixed fault categories, real-world machinery often exhibits even more complex failure mode combinations.
\end{itemize}

Despite these limitations, the CWRU dataset remains valuable for benchmarking and comparative studies due to its widespread adoption, well-documented characteristics, and availability. Our results on this dataset provide meaningful insights while acknowledging the need for validation on additional datasets for comprehensive generalization assessment.


\section{Methodology}
\label{sec:methodology}

\subsection{CNN-Based Approach (Milestone 1)}
\label{sec:methodology_cnn}
Convolutional Neural Networks have emerged as powerful tools for automatic feature learning from raw data. This section details our CNN-based approach for bearing fault diagnosis, covering architectural design, implementation choices, and the rationale behind our methodology.

\subsubsection{CNN Fundamentals for 1D Signals}

Traditional CNNs were designed for 2D image data, but the core principles transfer naturally to 1D time-series data. A 1D convolutional layer applies learnable filters across the temporal dimension:

\begin{equation}
y[n] = \sum_{k=0}^{K-1} w[k] \cdot x[n-k] + b
\end{equation}

where $w$ represents the filter weights, $x$ is the input signal, $K$ is the kernel size, and $b$ is the bias term. Multiple filters are applied in parallel, each learning to detect different patterns in the signal.

The key advantages of CNNs for vibration signal processing include:

\begin{itemize}
    \item \textbf{Local Pattern Detection:} Convolutional filters detect local patterns regardless of their position in the signal (translation invariance), which aligns well with the nature of bearing fault signatures that can occur at any point in the rotation cycle.

    \item \textbf{Hierarchical Feature Learning:} Stacking multiple convolutional layers enables hierarchical feature extraction, with early layers capturing basic patterns (impulses, oscillations) and deeper layers learning complex fault-specific signatures.

    \item \textbf{Parameter Efficiency:} Weight sharing across the temporal dimension dramatically reduces the number of parameters compared to fully-connected networks, preventing overfitting and enabling deeper architectures.

    \item \textbf{Spatial Hierarchy:} Pooling operations progressively reduce the temporal resolution while increasing the receptive field, allowing the network to capture both fine-grained and coarse-grained patterns.
\end{itemize}

\subsubsection{Architectural Variants}

We implement and evaluate 15+ CNN architectures spanning multiple design philosophies. This comprehensive exploration enables systematic assessment of architectural choices and their impact on fault diagnosis performance.

\paragraph{Basic CNN Architectures}

Our basic CNN implementation (CNN1D) serves as a baseline, consisting of:

\begin{itemize}
    \item 4 convolutional blocks, each containing:
    \begin{itemize}
        \item 1D convolutional layer (64, 128, 256, 512 filters respectively)
        \item Batch normalization for training stability
        \item ReLU activation function
        \item Max pooling for downsampling
    \end{itemize}
    \item Global average pooling to aggregate temporal information
    \item Fully-connected classification head
\end{itemize}

This architecture contains approximately 2.3M parameters and provides a straightforward baseline for comparison.

\paragraph{ResNet Family}

Residual Networks\cite{he2016deep} address the degradation problem in deep networks through skip connections. We adapt three ResNet variants for 1D signals:

\textbf{ResNet-18:} The shallowest variant with 18 layers, containing:
\begin{itemize}
    \item Initial conv layer: 64 filters, kernel size 7, stride 2
    \item 4 residual blocks with [2, 2, 2, 2] basic residual units
    \item Filter progression: [64, 128, 256, 512]
    \item Approximately 11M parameters
\end{itemize}

\textbf{ResNet-34:} Medium-depth variant with 34 layers:
\begin{itemize}
    \item Same initial conv layer as ResNet-18
    \item 4 residual blocks with [3, 4, 6, 3] basic residual units
    \item Filter progression: [64, 128, 256, 512]
    \item Approximately 21M parameters
\end{itemize}

\textbf{ResNet-50:} Deepest variant with 50 layers:
\begin{itemize}
    \item Initial conv layer: 64 filters, kernel size 7, stride 2
    \item 4 residual blocks with [3, 4, 6, 3] bottleneck residual units
    \item Bottleneck design uses 1×1 convolutions to reduce/restore dimensionality
    \item Filter progression: [256, 512, 1024, 2048]
    \item Approximately 25M parameters
\end{itemize}

Each residual unit implements the identity mapping:

\begin{equation}
\mathbf{y} = \mathcal{F}(\mathbf{x}, \{W_i\}) + \mathbf{x}
\end{equation}

where $\mathcal{F}$ represents the residual function learned by stacked convolutional layers, and the skip connection adds the input $\mathbf{x}$ directly to the output. This formulation enables gradient flow through hundreds of layers and has proven highly effective for image classification.

\paragraph{EfficientNet Family}

EfficientNets\cite{tan2019efficientnet} achieve state-of-the-art performance through compound scaling that uniformly scales network width, depth, and resolution. We adapt three EfficientNet variants:

\textbf{EfficientNet-B0:} The base model with:
\begin{itemize}
    \item 7 Mobile Inverted Bottleneck (MBConv) blocks
    \item Squeeze-and-Excitation (SE) attention modules
    \item Swish activation function: $f(x) = x \cdot \sigma(\beta x)$
    \item Approximately 5.3M parameters
    \item Optimized width/depth multipliers
\end{itemize}

\textbf{EfficientNet-B2:} Scaled version of B0:
\begin{itemize}
    \item Deeper network (increased depth multiplier)
    \item Wider layers (increased width multiplier)
    \item Approximately 9.2M parameters
    \item Maintains computational efficiency through efficient convolutions
\end{itemize}

\textbf{EfficientNet-B4:} Larger scaled version:
\begin{itemize}
    \item Further increased depth and width
    \item Approximately 19M parameters
    \item Higher capacity for complex pattern learning
\end{itemize}

The MBConv blocks in EfficientNet use depthwise separable convolutions and expansion layers, significantly reducing computational cost while maintaining representational power. SE modules add channel-wise attention, allowing the network to emphasize informative features:

\begin{equation}
\mathbf{y} = \mathbf{x} \odot \sigma(W_2 \delta(W_1 \text{GAP}(\mathbf{x})))
\end{equation}

where GAP denotes global average pooling, $W_1$ and $W_2$ are fully-connected layers, $\delta$ is ReLU activation, $\sigma$ is sigmoid activation, and $\odot$ represents element-wise multiplication.

\subsubsection{Input Representation}

Raw vibration signals are represented as 1D tensors of shape $[B, 1, L]$ where:
\begin{itemize}
    \item $B$: Batch size (typically 32 or 64)
    \item $1$: Number of channels (single-channel accelerometer data)
    \item $L$: Signal length (102,400 samples)
\end{itemize}

This representation treats the vibration signal as a single-channel "image" in the time domain, allowing direct application of convolutional operations without transformation to frequency or time-frequency domains.

\subsubsection{Network Components}

\paragraph{Convolutional Layers}

1D convolutional layers extract local patterns from the input signal. We systematically vary:

\begin{itemize}
    \item \textbf{Kernel Size:} Controls the receptive field of filters. Smaller kernels (3, 5) capture fine-grained patterns, while larger kernels (7, 11) detect broader structures. We primarily use kernel sizes of 3 and 7.

    \item \textbf{Number of Filters:} Determines the representational capacity. We follow the common practice of progressively increasing filter counts in deeper layers: 64 → 128 → 256 → 512.

    \item \textbf{Stride:} Controls the downsampling rate. We use stride 2 in specific layers to reduce temporal resolution and computational cost.

    \item \textbf{Padding:} We apply "same" padding to preserve temporal dimensions and "valid" padding for dimension reduction.
\end{itemize}

\paragraph{Activation Functions}

Non-linear activation functions enable networks to learn complex decision boundaries:

\begin{itemize}
    \item \textbf{ReLU} (Rectified Linear Unit): $f(x) = \max(0, x)$

    Used in ResNet and basic CNN architectures for its simplicity and effective gradient propagation.

    \item \textbf{Swish}: $f(x) = x \cdot \sigma(\beta x)$

    Used in EfficientNet architectures, providing smooth non-linearity and improved performance.
\end{itemize}

\paragraph{Normalization}

Batch normalization normalizes layer inputs across the mini-batch:

\begin{equation}
\hat{x} = \frac{x - \mu_{\mathcal{B}}}{\sqrt{\sigma_{\mathcal{B}}^2 + \epsilon}}
\end{equation}

where $\mu_{\mathcal{B}}$ and $\sigma_{\mathcal{B}}^2$ are the batch mean and variance. Learnable scale ($\gamma$) and shift ($\beta$) parameters allow the network to adapt the normalization:

\begin{equation}
y = \gamma \hat{x} + \beta
\end{equation}

Batch normalization accelerates training, enables higher learning rates, and provides regularization through mini-batch statistics.

\paragraph{Pooling Operations}

Pooling layers reduce spatial dimensions and provide translation invariance:

\begin{itemize}
    \item \textbf{Max Pooling:} $y[n] = \max_{k \in \mathcal{N}(n)} x[k]$

    Selects the maximum value within each pooling window, emphasizing salient features.

    \item \textbf{Average Pooling:} $y[n] = \frac{1}{|\mathcal{N}(n)|} \sum_{k \in \mathcal{N}(n)} x[k]$

    Computes the mean within each window, providing smoother downsampling.

    \item \textbf{Global Average Pooling (GAP):} Aggregates the entire temporal dimension into a single value per channel, reducing parameters in the classification head.
\end{itemize}

\paragraph{Classification Head}

After convolutional feature extraction, we employ fully-connected layers for classification:

\begin{itemize}
    \item Global average pooling reduces feature maps to a fixed-size vector
    \item Optional dropout layer (p=0.5) for regularization
    \item Fully-connected layer mapping to 11 output classes
    \item Softmax activation for probability distribution:
    \begin{equation}
    p_i = \frac{e^{z_i}}{\sum_{j=1}^{11} e^{z_j}}
    \end{equation}
    where $z_i$ is the logit for class $i$.
\end{itemize}

\subsubsection{Training Methodology}

\paragraph{Loss Function}

We employ cross-entropy loss for multi-class classification:

\begin{equation}
\mathcal{L}_{CE} = -\frac{1}{N} \sum_{i=1}^{N} \sum_{c=1}^{C} y_{i,c} \log(\hat{y}_{i,c})
\end{equation}

where $N$ is the batch size, $C=11$ is the number of classes, $y_{i,c}$ is the one-hot encoded ground truth, and $\hat{y}_{i,c}$ is the predicted probability for sample $i$ and class $c$.

\paragraph{Optimization}

We use the Adam optimizer\cite{kingma2014adam} with adaptive learning rates:

\begin{align}
m_t &= \beta_1 m_{t-1} + (1-\beta_1) g_t \\
v_t &= \beta_2 v_{t-1} + (1-\beta_2) g_t^2 \\
\theta_t &= \theta_{t-1} - \alpha \frac{\hat{m}_t}{\sqrt{\hat{v}_t} + \epsilon}
\end{align}

where $g_t$ is the gradient at step $t$, $m_t$ and $v_t$ are first and second moment estimates, and $\beta_1=0.9$, $\beta_2=0.999$ are decay rates. Initial learning rate $\alpha=0.001$.

\paragraph{Learning Rate Scheduling}

We employ cosine annealing\cite{loshchilov2016sgdr} for smooth learning rate decay:

\begin{equation}
\eta_t = \eta_{min} + \frac{1}{2}(\eta_{max} - \eta_{min})\left(1 + \cos\left(\frac{T_{cur}}{T_{max}}\pi\right)\right)
\end{equation}

where $\eta_{max}=0.001$, $\eta_{min}=0.00001$, $T_{cur}$ is the current epoch, and $T_{max}$ is the total number of epochs.

\paragraph{Regularization Techniques}

\begin{enumerate}
    \item \textbf{Dropout:} Randomly drops units during training with probability $p=0.5$ in fully-connected layers, preventing co-adaptation of features.

    \item \textbf{Weight Decay:} L2 regularization penalizes large weights:
    \begin{equation}
    \mathcal{L}_{total} = \mathcal{L}_{CE} + \lambda \sum_{i} w_i^2
    \end{equation}
    with $\lambda=0.0001$.

    \item \textbf{Data Augmentation:} Applied stochastically during training as described in Section~\ref{sec:dataset}.

    \item \textbf{Early Stopping:} Training halts if validation loss doesn't improve for 15 consecutive epochs, preventing overfitting.
\end{enumerate}

\paragraph{Mixed Precision Training}

We leverage automatic mixed precision (AMP) training with FP16 arithmetic for computational efficiency:

\begin{itemize}
    \item Forward passes computed in FP16 for speed
    \item Backward passes use FP32 for numerical stability
    \item Gradient scaling prevents underflow
    \item Achieves ~2× speedup on modern GPUs with minimal accuracy impact
\end{itemize}

\subsubsection{Implementation Details}

All CNN models are implemented in PyTorch 2.0.1 using the following configuration:

\begin{lstlisting}[language=Python]
# Model Configuration
num_classes = 11
input_shape = (1, 102400)
batch_size = 32

# Training Configuration
num_epochs = 75
initial_lr = 0.001
optimizer = 'adam'
scheduler = 'cosine'
weight_decay = 0.0001

# Regularization
dropout_prob = 0.5
data_augmentation = True
early_stopping_patience = 15

# Hardware
device = 'cuda'  # NVIDIA GPU
mixed_precision = True
num_workers = 4  # Data loading threads
\end{lstlisting}

\subsubsection{Model Selection and Evaluation}

During training, we monitor both training and validation metrics. The model checkpoint with the highest validation accuracy is selected as the final model for test set evaluation. This prevents selecting models that overfit to the training data.

For each architecture, we report:
\begin{itemize}
    \item Overall classification accuracy
    \item Per-class precision, recall, and F1-score
    \item Confusion matrix
    \item Number of parameters
    \item Model size (MB)
    \item Inference time per sample
    \item Training time per epoch
\end{itemize}

\subsubsection{Rationale for Multiple Architectures}

The inclusion of 15+ CNN variants serves several purposes:

\begin{enumerate}
    \item \textbf{Baseline Establishment:} Simple CNNs provide performance baselines for comparison with more sophisticated architectures.

    \item \textbf{Architectural Comparison:} Systematic evaluation reveals which architectural innovations (residual connections, efficient scaling, attention mechanisms) provide the most benefit for bearing fault diagnosis.

    \item \textbf{Computational Trade-offs:} Different architectures offer various accuracy-efficiency trade-offs, allowing selection based on deployment constraints (edge devices vs. cloud servers).

    \item \textbf{Robustness Assessment:} Evaluating multiple architectures provides confidence that results are not architecture-specific artifacts.

    \item \textbf{Transfer Learning Potential:} Understanding which architectures excel enables informed choices for transfer learning to new bearing types or fault categories.
\end{enumerate}

\subsubsection{Expected Outcomes}

Based on literature and preliminary experiments, we hypothesize that:

\begin{itemize}
    \item ResNet architectures will achieve high accuracy due to their depth and skip connections
    \item EfficientNet models will provide strong performance with fewer parameters
    \item Deeper networks (ResNet-50, EfficientNet-B4) may risk overfitting given the dataset size
    \item Basic CNNs will establish competitive baselines, demonstrating the power of automatic feature learning
\end{itemize}

The actual performance results are presented in Section~\ref{sec:results}, where we provide comprehensive empirical validation of these hypotheses.


\subsection{LSTM-Based Approach (Milestone 2)}
\label{sec:methodology_lstm}
While CNNs excel at spatial pattern recognition, bearing vibration signals are fundamentally temporal sequences where the ordering and evolution of patterns contain critical diagnostic information. This section presents our LSTM-based approach, which explicitly models temporal dependencies in vibration data.

\subsubsection{Motivation for Temporal Modeling}

Bearing faults manifest as time-varying phenomena in vibration signals. Several temporal characteristics are diagnostically relevant:

\begin{itemize}
    \item \textbf{Sequential Patterns:} Fault signatures often appear as specific temporal sequences of impulses or oscillations that unfold over multiple rotation cycles.

    \item \textbf{Long-Range Dependencies:} Early fault indicators may correlate with features appearing seconds later in the signal, requiring models that can remember information across extended time spans.

    \item \textbf{Temporal Evolution:} Fault progression changes vibration patterns over time, and the rate and nature of these changes provide diagnostic clues.

    \item \textbf{Phase Relationships:} The relative timing and phase between different signal components reveals fault mechanisms and locations.
\end{itemize}

Traditional CNNs, while capable of learning local temporal patterns through convolution, do not explicitly model long-range temporal dependencies or maintain internal state across time steps. Recurrent architectures address these limitations.

\subsubsection{LSTM Fundamentals}

Long Short-Term Memory networks\cite{hochreiter1997long} are a specialized form of Recurrent Neural Networks designed to capture long-term dependencies while avoiding the vanishing gradient problem that plagues standard RNNs.

\paragraph{LSTM Cell Architecture}

An LSTM cell maintains two state vectors: the cell state $\mathbf{c}_t$ (long-term memory) and the hidden state $\mathbf{h}_t$ (short-term output). At each time step $t$, the cell processes input $\mathbf{x}_t$ and previous hidden state $\mathbf{h}_{t-1}$ through three gates and one cell update:

\textbf{Forget Gate:} Determines what information to discard from cell state:
\begin{equation}
\mathbf{f}_t = \sigma(W_f \mathbf{x}_t + U_f \mathbf{h}_{t-1} + \mathbf{b}_f)
\end{equation}

\textbf{Input Gate:} Determines what new information to store:
\begin{equation}
\mathbf{i}_t = \sigma(W_i \mathbf{x}_t + U_i \mathbf{h}_{t-1} + \mathbf{b}_i)
\end{equation}

\textbf{Candidate Cell State:} Proposes new information to add:
\begin{equation}
\tilde{\mathbf{c}}_t = \tanh(W_c \mathbf{x}_t + U_c \mathbf{h}_{t-1} + \mathbf{b}_c)
\end{equation}

\textbf{Cell State Update:} Combines forget and input gates:
\begin{equation}
\mathbf{c}_t = \mathbf{f}_t \odot \mathbf{c}_{t-1} + \mathbf{i}_t \odot \tilde{\mathbf{c}}_t
\end{equation}

\textbf{Output Gate:} Determines what to output based on cell state:
\begin{equation}
\mathbf{o}_t = \sigma(W_o \mathbf{x}_t + U_o \mathbf{h}_{t-1} + \mathbf{b}_o)
\end{equation}

\textbf{Hidden State:} Final output of the cell:
\begin{equation}
\mathbf{h}_t = \mathbf{o}_t \odot \tanh(\mathbf{c}_t)
\end{equation}

where $\sigma$ denotes the sigmoid function, $\tanh$ is the hyperbolic tangent, $\odot$ represents element-wise multiplication, and $W$, $U$, $\mathbf{b}$ are learnable parameters.

The gating mechanism allows LSTMs to selectively retain, forget, or update information over long sequences, addressing the vanishing gradient problem through the cell state's additive update structure.

\subsubsection{LSTM Variants}

We implement and evaluate two LSTM variants that differ in their temporal processing strategy:

\paragraph{Vanilla LSTM (Unidirectional)}

The standard LSTM processes sequences in a single direction (forward in time):

\begin{equation}
\mathbf{h}_t = \text{LSTM}(\mathbf{x}_t, \mathbf{h}_{t-1}, \mathbf{c}_{t-1})
\end{equation}

This architecture is suitable for causal applications where only past information is available (e.g., real-time monitoring). Our implementation consists of:

\begin{itemize}
    \item Input dimension: 1 (single-channel vibration signal)
    \item Hidden dimension: 128 (default) or 256
    \item Number of layers: 2 stacked LSTM layers
    \item Dropout: 0.5 between LSTM layers
    \item Parameters: Approximately 200K for hidden size 128
\end{itemize}

The stacked architecture allows the second LSTM layer to learn higher-level temporal abstractions from the first layer's outputs.

\paragraph{Bidirectional LSTM (BiLSTM)}

Bidirectional LSTMs\cite{graves2005framewise} process sequences in both forward and backward directions, providing complete temporal context:

\begin{align}
\overrightarrow{\mathbf{h}}_t &= \text{LSTM}_{forward}(\mathbf{x}_t, \overrightarrow{\mathbf{h}}_{t-1}, \overrightarrow{\mathbf{c}}_{t-1}) \\
\overleftarrow{\mathbf{h}}_t &= \text{LSTM}_{backward}(\mathbf{x}_t, \overleftarrow{\mathbf{h}}_{t+1}, \overleftarrow{\mathbf{c}}_{t+1}) \\
\mathbf{h}_t &= [\overrightarrow{\mathbf{h}}_t; \overleftarrow{\mathbf{h}}_t]
\end{align}

where $[\cdot; \cdot]$ denotes concatenation. The bidirectional architecture doubles the output dimension and parameter count:

\begin{itemize}
    \item Input dimension: 1
    \item Hidden dimension: 128 (per direction)
    \item Output dimension: 256 (concatenated forward and backward)
    \item Number of layers: 2 stacked BiLSTM layers
    \item Dropout: 0.5 between layers
    \item Parameters: Approximately 400K for hidden size 128
\end{itemize}

BiLSTMs are appropriate for offline analysis where the entire signal is available before classification. The ability to incorporate future context often improves classification performance at the cost of doubled computation.

\subsubsection{Signal Preprocessing for LSTMs}

Unlike CNNs that process the entire signal as a single input, LSTMs require sequential input. We adopt two preprocessing strategies:

\paragraph{Direct Sequence Input}

The raw vibration signal of length $L=102{,}400$ is treated as a sequence of $L$ time steps:

\begin{equation}
\mathbf{X} = [\mathbf{x}_1, \mathbf{x}_2, \ldots, \mathbf{x}_L] \in \mathbb{R}^{L \times 1}
\end{equation}

where each time step contains a single scalar value. This representation preserves the finest temporal granularity but results in very long sequences that are computationally expensive and may suffer from gradient vanishing despite LSTM's gating mechanisms.

\paragraph{Windowed Sequence Input}

Alternatively, we segment the signal into non-overlapping windows and extract features from each window:

\begin{equation}
\mathbf{X} = [\mathbf{x}_1, \mathbf{x}_2, \ldots, \mathbf{x}_T] \in \mathbb{R}^{T \times D}
\end{equation}

where $T = L / w$ is the number of windows of size $w$, and $D$ is the feature dimension per window (e.g., mean, standard deviation, max, min). This reduces sequence length at the cost of reduced temporal resolution.

For our experiments, we primarily use direct sequence input for sequences up to 10,240 samples (downsampled by a factor of 10) to balance temporal resolution with computational efficiency. Longer sequences would require excessive memory and training time.

\subsubsection{Network Architecture}

Our LSTM-based models follow this general architecture:

\begin{enumerate}
    \item \textbf{Input Layer:} Receives signal of shape $[B, T, 1]$ where $B$ is batch size and $T$ is sequence length.

    \item \textbf{LSTM Layers:} Two stacked LSTM (or BiLSTM) layers with hidden size $h$:
    \begin{itemize}
        \item Layer 1: Processes input sequence, outputs hidden states of dimension $h$ (or $2h$ for BiLSTM)
        \item Dropout (0.5) for regularization
        \item Layer 2: Processes Layer 1 outputs, outputs hidden states
        \item Dropout (0.5)
    \end{itemize}

    \item \textbf{Temporal Aggregation:} Multiple strategies for combining temporal outputs:
    \begin{itemize}
        \item \textbf{Last Hidden State:} Use final time step $\mathbf{h}_T$
        \item \textbf{Mean Pooling:} Average over all time steps $\frac{1}{T}\sum_{t=1}^T \mathbf{h}_t$
        \item \textbf{Max Pooling:} Maximum over time dimension
        \item \textbf{Attention:} Learnable weighted combination
    \end{itemize}

    \item \textbf{Classification Head:}
    \begin{itemize}
        \item Fully-connected layer mapping aggregated features to 11 classes
        \item Softmax activation for probability distribution
    \end{itemize}
\end{enumerate}

\subsubsection{Temporal Attention Mechanism}

For enhanced performance, we optionally incorporate an attention mechanism that learns to weight different time steps according to their diagnostic relevance:

\begin{align}
e_t &= \mathbf{v}^T \tanh(W \mathbf{h}_t + \mathbf{b}) \\
\alpha_t &= \frac{\exp(e_t)}{\sum_{t'=1}^T \exp(e_{t'})} \\
\mathbf{h}_{att} &= \sum_{t=1}^T \alpha_t \mathbf{h}_t
\end{align}

where $e_t$ is the attention score for time step $t$, $\alpha_t$ is the normalized attention weight, and $\mathbf{h}_{att}$ is the attention-weighted aggregated hidden state. The parameters $\mathbf{v}$, $W$, and $\mathbf{b}$ are learned during training.

Attention provides interpretability by revealing which temporal regions the model considers important for classification. High attention weights on specific time segments indicate fault-relevant patterns.

\subsubsection{Training Methodology}

\paragraph{Loss Function and Optimization}

Similar to the CNN approach, we use cross-entropy loss with Adam optimizer. However, LSTM training presents unique challenges:

\begin{itemize}
    \item \textbf{Gradient Clipping:} We apply gradient clipping with threshold 5.0 to prevent exploding gradients:
    \begin{equation}
    \mathbf{g} \leftarrow \min\left(1, \frac{\theta}{||\mathbf{g}||}\right) \mathbf{g}
    \end{equation}
    where $\theta=5.0$ is the clipping threshold.

    \item \textbf{Learning Rate:} We use a lower initial learning rate (0.001 to 0.0005) for LSTM training compared to CNNs, as RNNs can be more sensitive to learning rate choices.

    \item \textbf{Batch Size:} Smaller batch sizes (16 or 32) are used due to the memory requirements of processing long sequences.
\end{itemize}

\paragraph{Regularization}

LSTM overfitting is prevented through:

\begin{enumerate}
    \item \textbf{Dropout:} Applied between LSTM layers (not within LSTM cells) with probability 0.5

    \item \textbf{Recurrent Dropout:} Optionally applied to recurrent connections:
    \begin{equation}
    \mathbf{h}_t = \text{LSTM}(\mathbf{x}_t, \mathbf{D} \odot \mathbf{h}_{t-1})
    \end{equation}
    where $\mathbf{D}$ is a dropout mask shared across time steps

    \item \textbf{Weight Decay:} L2 regularization with coefficient $\lambda=0.0001$

    \item \textbf{Early Stopping:} Validation-based early stopping with patience of 15 epochs
\end{enumerate}

\subsubsection{Computational Considerations}

LSTM training is computationally intensive due to sequential processing that prevents parallelization across time steps. Key computational challenges include:

\begin{itemize}
    \item \textbf{Memory:} Backpropagation through time (BPTT) requires storing hidden states for all time steps, leading to $\mathcal{O}(T)$ memory complexity

    \item \textbf{Time:} Sequential processing results in slower training compared to CNNs that can parallelize across the signal length

    \item \textbf{Sequence Length:} Longer sequences exponentially increase both memory and computation requirements
\end{itemize}

To mitigate these challenges, we:
\begin{itemize}
    \item Downsample long signals to reduce sequence length
    \item Use smaller batch sizes to fit in GPU memory
    \item Leverage cuDNN-optimized LSTM implementations for efficient GPU execution
\end{itemize}

\subsubsection{Implementation Details}

LSTM models are implemented in PyTorch 2.0.1:

\begin{lstlisting}[language=Python]
# Model Configuration
num_classes = 11
sequence_length = 10240  # Downsampled from 102400
hidden_size = 128  # LSTM hidden dimension
num_layers = 2
bidirectional = True  # For BiLSTM

# Training Configuration
num_epochs = 75
batch_size = 32
initial_lr = 0.001
optimizer = 'adam'
scheduler = 'cosine'

# Regularization
dropout = 0.5
gradient_clip = 5.0
weight_decay = 0.0001

# Computational
device = 'cuda'
cudnn_benchmark = True  # Optimize cuDNN for fixed input sizes
\end{lstlisting}

\subsubsection{Comparison: Vanilla LSTM vs. BiLSTM}

Table~\ref{tab:lstm_comparison} provides a theoretical comparison of the two LSTM variants:

\begin{table}[h]
\centering
\caption{Comparison of LSTM Variants}
\label{tab:lstm_comparison}
\begin{tabular}{@{}lcc@{}}
\toprule
\textbf{Characteristic} & \textbf{Vanilla LSTM} & \textbf{BiLSTM} \\
\midrule
Processing Direction & Forward only & Forward + Backward \\
Temporal Context & Past only & Past + Future \\
Output Dimension & $h$ & $2h$ \\
Parameters & $\sim$200K & $\sim$400K \\
Computational Cost & 1× & 2× \\
Real-time Capable & Yes & No \\
Typical Use Case & Online monitoring & Offline analysis \\
Expected Accuracy & \todo{TBD\%} & \todo{TBD\%} \\
\bottomrule
\end{tabular}
\end{table}

We hypothesize that BiLSTM will achieve higher accuracy due to complete temporal context, while Vanilla LSTM offers computational efficiency suitable for real-time applications.

\subsubsection{Expected Outcomes}

Based on the temporal nature of bearing vibration data, we expect LSTM-based approaches to:

\begin{itemize}
    \item Effectively capture sequential patterns and temporal dependencies
    \item Demonstrate complementary strengths to CNN approaches
    \item Show particular advantage for fault types with strong temporal signatures
    \item Achieve competitive accuracy despite having fewer parameters than large CNNs
    \item Potentially struggle with very long sequence lengths due to gradient issues
\end{itemize}

The empirical validation of these expectations is presented in Section~\ref{sec:results}, where we provide comprehensive performance analysis and comparison with CNN-based approaches.


\subsection{Hybrid CNN-LSTM Approach (Milestone 3)}
\label{sec:methodology_hybrid}
The complementary nature of CNNs (spatial pattern recognition) and LSTMs (temporal modeling) motivates their integration into hybrid architectures. This section presents our novel configurable hybrid framework that systematically combines CNN feature extraction with LSTM temporal modeling.

\subsubsection{Motivation for Hybrid Architectures}

Bearing vibration signals exhibit both spatial and temporal characteristics that are diagnostically relevant:

\begin{itemize}
    \item \textbf{Spatial Characteristics:} Local impulses, frequency components, signal textures, and fault-specific signatures that CNNs excel at detecting

    \item \textbf{Temporal Characteristics:} Sequential patterns, phase relationships, signal evolution over time, and long-range correlations that LSTMs capture effectively
\end{itemize}

Neither pure CNN nor pure LSTM approaches fully exploit both dimensions. A hybrid architecture can potentially achieve superior performance by:

\begin{enumerate}
    \item Using CNNs as automatic feature extractors, eliminating manual feature engineering
    \item Processing CNN-extracted features through LSTMs to model their temporal evolution
    \item Learning end-to-end from raw signals to fault labels
    \item Leveraging the strengths of both architectures while mitigating their individual limitations
\end{enumerate}

\subsubsection{Configurable Hybrid Framework}

Unlike prior hybrid approaches that tightly couple specific CNN and LSTM architectures, we introduce a \textit{configurable framework} that allows arbitrary combinations. This design philosophy provides several advantages:

\begin{itemize}
    \item \textbf{Flexibility:} Any CNN backbone can be paired with any LSTM type
    \item \textbf{Systematic Exploration:} Enables principled evaluation of different combinations
    \item \textbf{Application-Specific Optimization:} Allows selection based on accuracy, efficiency, or deployment constraints
    \item \textbf{Research Platform:} Facilitates investigation of which architectural choices matter most
\end{itemize}

Our framework supports:
\begin{itemize}
    \item \textbf{7 CNN Backbones:} cnn1d, resnet18, resnet34, resnet50, efficientnet\_b0, efficientnet\_b2, efficientnet\_b4
    \item \textbf{2 LSTM Types:} Vanilla LSTM, Bidirectional LSTM
    \item \textbf{4 Pooling Methods:} Mean, max, last timestep, attention
    \item \textbf{Total Combinations:} 7 × 2 × 4 = 56 possible configurations (with variations in LSTM hidden size and layers, the space exceeds 100 configurations)
\end{itemize}

\subsubsection{Architectural Design}

\paragraph{Overall Architecture Flow}

The hybrid architecture follows a clear hierarchical processing pipeline:

\begin{equation}
\text{Raw Signal} \xrightarrow{\text{CNN}} \text{Feature Sequence} \xrightarrow{\text{LSTM}} \text{Temporal Features} \xrightarrow{\text{Pool}} \text{Fixed Vector} \xrightarrow{\text{FC}} \text{Classification}
\end{equation}

More precisely:

\begin{enumerate}
    \item \textbf{Input:} Raw vibration signal $\mathbf{x} \in \mathbb{R}^{1 \times L}$ where $L=102{,}400$

    \item \textbf{CNN Feature Extraction:} CNN backbone processes the signal through multiple convolutional and pooling layers, producing a feature sequence:
    \begin{equation}
    \mathbf{F} = \text{CNN}(\mathbf{x}) \in \mathbb{R}^{T \times D}
    \end{equation}
    where $T$ is the temporal length after CNN processing, and $D$ is the feature dimension

    \item \textbf{LSTM Temporal Modeling:} The feature sequence is processed by LSTM layers:
    \begin{equation}
    \mathbf{H} = \text{LSTM}(\mathbf{F}) \in \mathbb{R}^{T \times H}
    \end{equation}
    where $H$ is the LSTM hidden size (or $2H$ for BiLSTM)

    \item \textbf{Temporal Aggregation:} The LSTM output sequence is pooled to a fixed-size vector:
    \begin{equation}
    \mathbf{h}_{agg} = \text{Pool}(\mathbf{H}) \in \mathbb{R}^{H}
    \end{equation}

    \item \textbf{Classification:} Fully-connected layers map to class probabilities:
    \begin{equation}
    \mathbf{p} = \text{softmax}(W \mathbf{h}_{agg} + \mathbf{b}) \in \mathbb{R}^{11}
    \end{equation}
\end{enumerate}

\paragraph{CNN Backbone Adaptation}

To use pretrained CNN models as feature extractors, we modify their architecture:

\begin{enumerate}
    \item \textbf{Remove Classification Head:} The final fully-connected layers designed for end-to-end classification are removed

    \item \textbf{Extract Feature Maps:} We extract activations from the final convolutional layer before global pooling, preserving spatial/temporal structure

    \item \textbf{Feature Dimension:} The CNN output dimension $D$ varies by architecture:
    \begin{itemize}
        \item ResNet-18: $D=512$
        \item ResNet-34: $D=512$
        \item ResNet-50: $D=2048$
        \item EfficientNet-B0: $D=1280$
        \item EfficientNet-B2: $D=1408$
        \item EfficientNet-B4: $D=1792$
    \end{itemize}

    \item \textbf{Temporal Length:} After multiple convolutional and pooling operations, the temporal dimension is significantly reduced. For example:
    \begin{itemize}
        \item Input: 102,400 samples
        \item After CNN: Typical $T \in [50, 200]$ depending on architecture and pooling strategies
    \end{itemize}
\end{enumerate}

The CNN backbone can be used in two modes:

\textbf{Frozen Mode:} CNN weights are fixed (not updated during training). This is useful when:
\begin{itemize}
    \item The CNN has been pretrained on similar data
    \item Computational resources are limited
    \item Only LSTM adaptation is desired
    \item Fast training is required
\end{itemize}

\textbf{Fine-tuning Mode:} CNN weights are updated end-to-end with LSTM. This is useful when:
\begin{itemize}
    \item Optimal performance is prioritized over training time
    \item Sufficient training data is available
    \item The task differs significantly from CNN pretraining
\end{itemize}

\paragraph{LSTM Processing Layer}

The LSTM processes the CNN feature sequence:

\begin{lstlisting}[language=Python]
class HybridCNNLSTM(nn.Module):
    def __init__(self, cnn_backbone, lstm_type='bilstm',
                 lstm_hidden_size=256, lstm_num_layers=2,
                 pooling_method='mean'):
        super().__init__()

        # CNN feature extractor
        self.cnn = self._prepare_cnn(cnn_backbone)
        self.cnn_output_dim = self._get_cnn_output_dim()

        # LSTM temporal modeling
        self.lstm = nn.LSTM(
            input_size=self.cnn_output_dim,
            hidden_size=lstm_hidden_size,
            num_layers=lstm_num_layers,
            batch_first=True,
            dropout=0.5 if lstm_num_layers > 1 else 0,
            bidirectional=(lstm_type == 'bilstm')
        )

        # Temporal pooling
        self.pooling = self._create_pooling(pooling_method)

        # Classification head
        lstm_output_dim = lstm_hidden_size * (2 if lstm_type == 'bilstm' else 1)
        self.classifier = nn.Linear(lstm_output_dim, num_classes)

    def forward(self, x):
        # CNN feature extraction: [B, 1, L] -> [B, T, D]
        features = self.cnn(x)

        # LSTM temporal modeling: [B, T, D] -> [B, T, H]
        lstm_out, _ = self.lstm(features)

        # Temporal pooling: [B, T, H] -> [B, H]
        pooled = self.pooling(lstm_out)

        # Classification: [B, H] -> [B, 11]
        logits = self.classifier(pooled)

        return logits
\end{lstlisting}

\paragraph{Temporal Pooling Strategies}

Four pooling methods aggregate LSTM outputs across the temporal dimension:

\textbf{Mean Pooling:}
\begin{equation}
\mathbf{h}_{mean} = \frac{1}{T} \sum_{t=1}^T \mathbf{h}_t
\end{equation}

Advantages: Stable, smooth aggregation of all temporal information. Disadvantages: Equally weights all time steps, potentially diluting important fault signatures.

\textbf{Max Pooling:}
\begin{equation}
\mathbf{h}_{max}[i] = \max_{t \in [1,T]} \mathbf{h}_t[i] \quad \forall i
\end{equation}

Advantages: Emphasizes salient features, robust to noise in irrelevant regions. Disadvantages: May be sensitive to outliers, discards timing information.

\textbf{Last Timestep:}
\begin{equation}
\mathbf{h}_{last} = \mathbf{h}_T
\end{equation}

Advantages: Simple, commonly used in sequence modeling. Disadvantages: Ignores most of the sequence, relies on LSTM's ability to compress all information into final state.

\textbf{Attention Pooling:}
\begin{align}
e_t &= \mathbf{v}^T \tanh(W_a \mathbf{h}_t + \mathbf{b}_a) \\
\alpha_t &= \text{softmax}(e_t) \\
\mathbf{h}_{att} &= \sum_{t=1}^T \alpha_t \mathbf{h}_t
\end{align}

Advantages: Learnable weighting, emphasizes diagnostically relevant time steps, interpretable. Disadvantages: Additional parameters, increased complexity.

\subsubsection{Recommended Configurations}

While our framework supports 56+ configurations, we provide three carefully designed recommendations optimized for different use cases:

\paragraph{Configuration 1: Best Accuracy}

\begin{itemize}
    \item \textbf{CNN Backbone:} ResNet-34
    \item \textbf{LSTM Type:} Bidirectional LSTM
    \item \textbf{LSTM Hidden Size:} 256
    \item \textbf{LSTM Layers:} 2
    \item \textbf{Pooling:} Mean pooling
    \item \textbf{Parameters:} \todo{$\sim$XX.X M}
    \item \textbf{Model Size:} \todo{$\sim$XXX MB}
    \item \textbf{Rationale:} ResNet-34 provides strong feature extraction with proven performance, BiLSTM captures full temporal context, mean pooling provides stable aggregation
\end{itemize}

\paragraph{Configuration 2: Best Efficiency}

\begin{itemize}
    \item \textbf{CNN Backbone:} EfficientNet-B2
    \item \textbf{LSTM Type:} Bidirectional LSTM
    \item \textbf{LSTM Hidden Size:} 256
    \item \textbf{LSTM Layers:} 2
    \item \textbf{Pooling:} Mean pooling
    \item \textbf{Parameters:} \todo{$\sim$XX.X M}
    \item \textbf{Model Size:} \todo{$\sim$XXX MB}
    \item \textbf{Rationale:} EfficientNet-B2 achieves excellent performance with fewer parameters and FLOPs, suitable for resource-constrained deployment
\end{itemize}

\paragraph{Configuration 3: Best Speed}

\begin{itemize}
    \item \textbf{CNN Backbone:} ResNet-18
    \item \textbf{LSTM Type:} Vanilla LSTM (unidirectional)
    \item \textbf{LSTM Hidden Size:} 128
    \item \textbf{LSTM Layers:} 2
    \item \textbf{Pooling:} Last timestep
    \item \textbf{Parameters:} \todo{$\sim$XX.X M}
    \item \textbf{Model Size:} \todo{$\sim$XXX MB}
    \item \textbf{Rationale:} ResNet-18 is the lightest ResNet, unidirectional LSTM halves computation, last timestep pooling is computationally cheapest
\end{itemize}

\subsubsection{Training Methodology}

Hybrid model training follows the same general protocol as CNNs and LSTMs, with specific considerations:

\paragraph{End-to-End Training}

The entire hybrid architecture is trained jointly:

\begin{equation}
\theta^* = \arg\min_\theta \mathcal{L}_{CE}(\text{Hybrid}_\theta(\mathbf{x}), y)
\end{equation}

where $\theta = \{\theta_{CNN}, \theta_{LSTM}, \theta_{classifier}\}$ encompasses all learnable parameters. Joint training allows the CNN to learn features optimized for subsequent LSTM processing, rather than generic features.

\paragraph{Learning Rate Strategy}

Due to different learning dynamics of CNN and LSTM components, we optionally employ discriminative learning rates:

\begin{itemize}
    \item CNN layers: $\eta_{CNN} = 0.0001$ (lower rate if using pretrained weights)
    \item LSTM layers: $\eta_{LSTM} = 0.001$ (higher rate for faster adaptation)
    \item Classifier: $\eta_{classifier} = 0.001$
\end{itemize}

However, for simplicity and unless otherwise noted, we use a uniform learning rate of 0.001 with cosine annealing.

\paragraph{Gradient Flow Considerations}

The hybrid architecture presents unique gradient flow challenges:

\begin{itemize}
    \item Gradients must backpropagate through both LSTM and CNN components
    \item LSTM gradients can be unstable, requiring gradient clipping
    \item Very deep CNNs may compound gradient issues
\end{itemize}

We address these through:
\begin{itemize}
    \item Gradient clipping (threshold = 5.0)
    \item Batch normalization in CNN layers
    \item Careful initialization of LSTM weights
    \item Monitoring gradient norms during training
\end{itemize}

\subsubsection{Computational Complexity Analysis}

The hybrid architecture's computational cost is the sum of CNN and LSTM costs:

\paragraph{Forward Pass Complexity}

\begin{align}
\text{FLOPs}_{total} &= \text{FLOPs}_{CNN} + \text{FLOPs}_{LSTM} \\
&\approx \mathcal{O}(K \cdot D_{in} \cdot D_{out} \cdot L) + \mathcal{O}(4 \cdot H^2 \cdot T \cdot N_{layers})
\end{align}

where $K$ is average kernel size, $D_{in}$, $D_{out}$ are layer dimensions, $H$ is LSTM hidden size, $T$ is sequence length after CNN, and $N_{layers}$ is number of LSTM layers.

\paragraph{Memory Complexity}

\begin{align}
\text{Memory}_{total} &= \text{Memory}_{params} + \text{Memory}_{activations} \\
&\approx \mathcal{O}(|\theta_{CNN}| + |\theta_{LSTM}|) + \mathcal{O}(B \cdot T \cdot H)
\end{align}

where $B$ is batch size. The activation memory for LSTM (required for backpropagation through time) can be substantial for long sequences.

\paragraph{Inference Time}

Inference time is dominated by the slower component. For typical configurations:
\begin{itemize}
    \item CNN forward pass: \todo{$\sim$XX ms}
    \item LSTM forward pass: \todo{$\sim$XX ms}
    \item Total per sample: \todo{$\sim$XX ms}
\end{itemize}

BiLSTM roughly doubles LSTM inference time compared to vanilla LSTM.

\subsubsection{Expected Performance Trade-offs}

We hypothesize the following trade-offs across hybrid configurations:

\begin{table}[h]
\centering
\caption{Expected Hybrid Configuration Trade-offs}
\label{tab:hybrid_tradeoffs}
\small
\begin{tabular}{@{}llll@{}}
\toprule
\textbf{Configuration} & \textbf{Accuracy} & \textbf{Speed} & \textbf{Memory} \\
\midrule
Recommended 1 (ResNet34+BiLSTM) & Highest & Moderate & High \\
Recommended 2 (EfficientNet-B2+BiLSTM) & High & Moderate & Moderate \\
Recommended 3 (ResNet18+LSTM) & Moderate-High & Highest & Low \\
\midrule
Custom (ResNet50+BiLSTM) & Very High & Slow & Very High \\
Custom (CNN1D+LSTM) & Moderate & Very Fast & Very Low \\
\bottomrule
\end{tabular}
\end{table}

\subsubsection{Advantages of Hybrid Approach}

The hybrid architecture offers several theoretical advantages:

\begin{enumerate}
    \item \textbf{Richer Representations:} Combines spatial patterns (CNN) with temporal dynamics (LSTM) for comprehensive feature learning

    \item \textbf{Hierarchical Processing:} CNN pre-processes raw signals into meaningful features, simplifying LSTM's task

    \item \textbf{Multi-Scale Analysis:} CNN captures local patterns, LSTM models global temporal structure

    \item \textbf{Flexibility:} Configurable framework enables optimization for specific requirements

    \item \textbf{Interpretability:} Attention mechanisms and feature visualizations provide insights into fault detection mechanisms
\end{enumerate}

\subsubsection{Implementation}

All hybrid models are implemented in PyTorch 2.0.1:

\begin{lstlisting}[language=Python]
# Example: Create hybrid model
from models import create_model

# Recommended configuration
model = create_model('recommended_1')

# Custom configuration
model = create_model(
    'custom',
    cnn_type='resnet34',
    lstm_type='bilstm',
    lstm_hidden_size=256,
    lstm_num_layers=2,
    pooling_method='mean',
    freeze_cnn=False
)

# Training configuration
optimizer = torch.optim.Adam(model.parameters(), lr=0.001)
scheduler = torch.optim.lr_scheduler.CosineAnnealingLR(
    optimizer, T_max=75
)
criterion = nn.CrossEntropyLoss()
\end{lstlisting}

\subsubsection{Research Questions}

Our hybrid framework enables investigation of several research questions:

\begin{enumerate}
    \item How do different CNN backbones affect hybrid performance when paired with the same LSTM?
    \item What is the performance gap between unidirectional and bidirectional LSTMs in the hybrid context?
    \item Which temporal pooling strategy yields the best performance?
    \item How does freezing CNN weights (transfer learning) impact accuracy and training time?
    \item What is the optimal balance between CNN and LSTM complexity?
\end{enumerate}

These questions are addressed empirically in Section~\ref{sec:results}, where we present comprehensive ablation studies and performance analysis. The configurable nature of our framework makes such systematic investigation tractable.


\section{Experimental Setup}
\label{sec:experimental}
This section details the experimental protocols, hardware infrastructure, software environment, evaluation metrics, and statistical analysis procedures used throughout this study.

\subsection{Hardware and Software Environment}

All experiments were conducted on consistent hardware to ensure fair comparison:

\paragraph{Hardware Configuration}
\begin{itemize}
    \item \textbf{GPU:} \todo{NVIDIA GPU model, XX GB VRAM}
    \item \textbf{CPU:} \todo{XX cores, XX GHz}
    \item \textbf{RAM:} \todo{XX GB}
    \item \textbf{Storage:} \todo{SSD configuration}
\end{itemize}

\paragraph{Software Stack}
\begin{itemize}
    \item \textbf{Operating System:} \todo{Linux distribution and version}
    \item \textbf{Python:} 3.9.12
    \item \textbf{PyTorch:} 2.0.1 with CUDA 11.8
    \item \textbf{cuDNN:} 8.7.0
    \item \textbf{NumPy:} 1.24.3
    \item \textbf{SciPy:} 1.10.1 (for .mat file loading)
    \item \textbf{scikit-learn:} 1.3.0 (for metrics)
    \item \textbf{Matplotlib:} 3.7.1 (for visualization)
\end{itemize}

\subsection{Training Configuration}

Unless otherwise specified, all models use the following training hyperparameters:

\paragraph{Common Training Parameters}
\begin{itemize}
    \item \textbf{Number of Epochs:} 75
    \item \textbf{Batch Size:} 32 for CNNs and Hybrids, 16-32 for LSTMs
    \item \textbf{Initial Learning Rate:} 0.001
    \item \textbf{Optimizer:} Adam with $\beta_1=0.9$, $\beta_2=0.999$, $\epsilon=10^{-8}$
    \item \textbf{Weight Decay:} 0.0001
    \item \textbf{Learning Rate Schedule:} Cosine annealing
    \item \textbf{Gradient Clipping:} Enabled for LSTMs and Hybrids (threshold = 5.0)
    \item \textbf{Mixed Precision:} Enabled (FP16) for computational efficiency
    \item \textbf{Random Seed:} 42 (for reproducibility)
\end{itemize}

\paragraph{Data Loading}
\begin{itemize}
    \item \textbf{Num Workers:} 4 (parallel data loading threads)
    \item \textbf{Pin Memory:} Enabled (for faster GPU transfer)
    \item \textbf{Shuffle:} Enabled for training set, disabled for validation/test sets
\end{itemize}

\paragraph{Regularization}
\begin{itemize}
    \item \textbf{Dropout:} 0.5 in fully-connected layers
    \item \textbf{Data Augmentation:} Applied to training set only
    \begin{itemize}
        \item Additive Gaussian noise: $\sigma \sim \mathcal{U}(0.01, 0.05)$
        \item Amplitude scaling: $\beta \sim \mathcal{U}(0.8, 1.2)$
        \item Temporal shifting: Random circular shift
        \item Application probability: 50\% per augmentation
    \end{itemize}
    \item \textbf{Early Stopping:} Patience of 15 epochs based on validation loss
\end{itemize}

\subsection{Model-Specific Configurations}

\paragraph{CNN Models}
\begin{itemize}
    \item Input shape: $[B, 1, 102400]$
    \item All 15+ architectures trained with identical hyperparameters
    \item Batch size: 32
    \item Global average pooling before classification
\end{itemize}

\paragraph{LSTM Models}
\begin{itemize}
    \item Input sequence length: 10,240 (downsampled by factor of 10)
    \item Hidden size: 128 or 256
    \item Number of layers: 2
    \item Bidirectional: True for BiLSTM, False for Vanilla LSTM
    \item Batch size: 32 (reduced to 16 if memory constrained)
\end{itemize}

\paragraph{Hybrid Models}
\begin{itemize}
    \item CNN backbone: Variable (resnet18, resnet34, resnet50, efficientnet variants)
    \item LSTM type: BiLSTM or Vanilla LSTM
    \item LSTM hidden size: 256 for recommended configs, variable for custom
    \item LSTM layers: 2
    \item Temporal pooling: Mean, max, last, or attention
    \item Batch size: 32
\end{itemize}

\subsection{Evaluation Metrics}

We employ comprehensive metrics to assess model performance from multiple perspectives:

\subsubsection{Classification Metrics}

\paragraph{Overall Accuracy}

The primary metric, measuring the percentage of correctly classified samples:

\begin{equation}
\text{Accuracy} = \frac{\text{Number of Correct Predictions}}{\text{Total Number of Samples}}
\end{equation}

\paragraph{Per-Class Metrics}

For each fault class $c$, we compute:

\textbf{Precision:} The fraction of predicted class $c$ samples that are truly class $c$:
\begin{equation}
\text{Precision}_c = \frac{TP_c}{TP_c + FP_c}
\end{equation}

\textbf{Recall (Sensitivity):} The fraction of true class $c$ samples correctly identified:
\begin{equation}
\text{Recall}_c = \frac{TP_c}{TP_c + FN_c}
\end{equation}

\textbf{F1-Score:} The harmonic mean of precision and recall:
\begin{equation}
F1_c = 2 \cdot \frac{\text{Precision}_c \cdot \text{Recall}_c}{\text{Precision}_c + \text{Recall}_c}
\end{equation}

where $TP_c$, $FP_c$, and $FN_c$ denote true positives, false positives, and false negatives for class $c$.

\paragraph{Macro-Averaged Metrics}

We compute macro-averaged precision, recall, and F1-score by averaging per-class metrics:

\begin{equation}
\text{Macro-F1} = \frac{1}{C} \sum_{c=1}^{C} F1_c
\end{equation}

where $C=11$ is the number of classes. Macro-averaging treats all classes equally, providing insight into performance across both common and rare faults.

\paragraph{Confusion Matrix}

We construct $11 \times 11$ confusion matrices showing the distribution of predictions for each true class. The confusion matrix enables identification of:
\begin{itemize}
    \item Frequently confused fault pairs
    \item Classes with high/low classification accuracy
    \item Systematic misclassification patterns
\end{itemize}

\subsubsection{Computational Metrics}

\paragraph{Model Size}

Total number of learnable parameters and model size in megabytes:
\begin{equation}
\text{Size (MB)} = \frac{\text{Number of Parameters} \times 4 \text{ bytes}}{1024^2}
\end{equation}

assuming 32-bit floating point storage.

\paragraph{Training Time}

\begin{itemize}
    \item Time per epoch (seconds)
    \item Total training time until convergence
    \item GPU memory usage during training
\end{itemize}

\paragraph{Inference Time}

\begin{itemize}
    \item Time per sample (milliseconds)
    \item Throughput (samples per second)
    \item Measured on both GPU and CPU for deployment considerations
\end{itemize}

\paragraph{FLOPs}

Floating point operations per forward pass, indicating computational complexity.

\subsubsection{Robustness Metrics}

\paragraph{Performance Under Noise}

To assess robustness, we evaluate models on test data corrupted with varying levels of additive Gaussian noise:

\begin{equation}
\mathbf{x}_{noisy} = \mathbf{x} + \mathcal{N}(0, \sigma^2)
\end{equation}

with $\sigma \in \{0.01, 0.05, 0.1, 0.2\}$, reporting accuracy degradation at each noise level.

\paragraph{Cross-Loading Generalization}

We evaluate models trained at one load condition on test data from different load conditions to assess generalization capability.

\subsection{Statistical Analysis}

To ensure robust conclusions, we employ rigorous statistical methodology:

\paragraph{Multiple Training Runs}

Each model configuration is trained \todo{3-5} times with different random seeds to account for training variability. We report:
\begin{itemize}
    \item Mean performance across runs
    \item Standard deviation
    \item Best and worst performance
\end{itemize}

\paragraph{Statistical Significance Testing}

When comparing two approaches, we apply paired t-tests with significance level $\alpha=0.05$. For multiple comparisons, we apply Bonferroni correction to control family-wise error rate.

\paragraph{Confidence Intervals}

We report 95\% confidence intervals for key metrics:
\begin{equation}
CI_{95} = \bar{x} \pm 1.96 \frac{s}{\sqrt{n}}
\end{equation}

where $\bar{x}$ is the sample mean, $s$ is the standard deviation, and $n$ is the number of runs.

\subsection{Experimental Protocols}

\subsubsection{Protocol 1: Individual Approach Evaluation}

For each of the three approaches (CNN, LSTM, Hybrid):

\begin{enumerate}
    \item Train all architectural variants with identical hyperparameters
    \item Evaluate on validation set to select best model per architecture
    \item Perform final evaluation on held-out test set
    \item Report comprehensive metrics (accuracy, precision, recall, F1, confusion matrix)
    \item Analyze failure cases and misclassification patterns
\end{enumerate}

\subsubsection{Protocol 2: Cross-Approach Comparison}

To fairly compare CNN, LSTM, and Hybrid approaches:

\begin{enumerate}
    \item Select best-performing architecture from each approach
    \item Ensure identical data splits, preprocessing, and evaluation procedures
    \item Train multiple times with different random seeds
    \item Compute mean and standard deviation of performance metrics
    \item Perform statistical significance testing
    \item Analyze computational trade-offs (accuracy vs. speed vs. model size)
\end{enumerate}

\subsubsection{Protocol 3: Ablation Studies}

For hybrid models, we conduct systematic ablation studies:

\begin{enumerate}
    \item \textbf{CNN Backbone Ablation:} Fix LSTM configuration, vary CNN backbone (resnet18 vs. resnet34 vs. efficientnet-b2, etc.)

    \item \textbf{LSTM Type Ablation:} Fix CNN backbone, compare Vanilla LSTM vs. BiLSTM

    \item \textbf{Pooling Method Ablation:} Fix CNN and LSTM, compare pooling strategies (mean vs. max vs. last vs. attention)

    \item \textbf{LSTM Depth Ablation:} Vary number of LSTM layers (1 vs. 2 vs. 3)

    \item \textbf{LSTM Hidden Size Ablation:} Vary hidden dimension (64 vs. 128 vs. 256 vs. 512)

    \item \textbf{Transfer Learning Ablation:} Compare frozen CNN vs. fine-tuned CNN
\end{enumerate}

Each ablation isolates one architectural choice while keeping others constant, enabling principled analysis of design decisions.

\subsection{Reproducibility Measures}

To ensure reproducibility, we implement:

\begin{enumerate}
    \item \textbf{Fixed Random Seeds:} All random number generators (Python, NumPy, PyTorch, cuDNN) are seeded with value 42

    \item \textbf{Deterministic Operations:} PyTorch deterministic mode enabled where possible:
    \begin{lstlisting}[language=Python]
torch.backends.cudnn.deterministic = True
torch.backends.cudnn.benchmark = False
    \end{lstlisting}

    \item \textbf{Version Pinning:} All package versions explicitly specified in requirements.txt

    \item \textbf{Complete Code Availability:} Full implementations, training scripts, and evaluation tools provided

    \item \textbf{Comprehensive Documentation:} Detailed README files, usage examples, and API documentation

    \item \textbf{Checkpoint Availability:} Trained model weights available for download (subject to storage constraints)
\end{enumerate}

\subsection{Validation Strategy}

We employ a three-level validation strategy:

\paragraph{Level 1: Validation Set During Training}

\begin{itemize}
    \item Monitor validation loss and accuracy after each epoch
    \item Early stopping based on validation performance
    \item Learning rate scheduling informed by validation plateau
\end{itemize}

\paragraph{Level 2: Validation Set for Model Selection}

\begin{itemize}
    \item Compare different architectures on validation set
    \item Select best hyperparameters using validation performance
    \item Ablation studies use validation set for intermediate comparisons
\end{itemize}

\paragraph{Level 3: Test Set for Final Evaluation}

\begin{itemize}
    \item Held-out test set used only once for final performance reporting
    \item No model selection or hyperparameter tuning based on test set
    \item Provides unbiased estimate of generalization performance
\end{itemize}

This hierarchical validation prevents information leakage and overfitting to evaluation data.

\subsection{Experimental Timeline}

The complete experimental program comprises:

\begin{itemize}
    \item \textbf{Milestone 1 (CNN):} \todo{XX} architectures $\times$ \todo{3-5} runs $\times$ \todo{XX} hours/run $\approx$ \todo{XX} hours total
    \item \textbf{Milestone 2 (LSTM):} \todo{XX} configurations $\times$ \todo{3-5} runs $\times$ \todo{XX} hours/run $\approx$ \todo{XX} hours total
    \item \textbf{Milestone 3 (Hybrid):} \todo{XX} configurations $\times$ \todo{3-5} runs $\times$ \todo{XX} hours/run $\approx$ \todo{XX} hours total
    \item \textbf{Ablation Studies:} \todo{XX} experiments $\times$ \todo{XX} hours/experiment $\approx$ \todo{XX} hours total
    \item \textbf{Total Compute Time:} \todo{XXX-XXX} GPU hours
\end{itemize}

All experiments were conducted \todo{over a period of XX weeks}.

\subsection{Ethical Considerations}

This research uses publicly available benchmark data (CWRU dataset) and does not involve human subjects, personal data, or sensitive information. The work aims to improve industrial safety through better fault diagnosis, presenting no ethical concerns. All software tools used are either open-source or properly licensed.


\section{Results and Discussion}
\label{sec:results}
This section presents comprehensive experimental results for all three approaches: CNN-based (Milestone 1), LSTM-based (Milestone 2), and Hybrid CNN-LSTM (Milestone 3). We provide quantitative performance metrics, qualitative analysis, computational comparisons, and ablation studies.

\subsection{Overall Performance Comparison}

Table~\ref{tab:overall_results} summarizes the performance of the best model from each approach on the test set.

\begin{table}[h]
\centering
\caption{Overall Performance Comparison Across Three Approaches}
\label{tab:overall_results}
\begin{tabular}{@{}lcccc@{}}
\toprule
\textbf{Approach} & \textbf{Best Model} & \textbf{Accuracy (\%)} & \textbf{Macro-F1} & \textbf{Parameters} \\
\midrule
CNN (M1) & \todo{ResNet-XX} & \todo{XX.XX $\pm$ X.XX} & \todo{X.XXX} & \todo{XX.XM} \\
LSTM (M2) & \todo{BiLSTM/Vanilla} & \todo{XX.XX $\pm$ X.XX} & \todo{X.XXX} & \todo{XX.XM} \\
Hybrid (M3) & \todo{Config-X} & \todo{XX.XX $\pm$ X.XX} & \todo{X.XXX} & \todo{XX.XM} \\
\bottomrule
\end{tabular}
\end{table}

\figplaceholder{Figure: Bar chart comparing overall accuracy of the three approaches with error bars showing standard deviation across multiple runs}

\paragraph{Key Findings:}

\todo{[After completing experiments, this paragraph will summarize which approach achieved the best performance, the magnitude of improvement, and whether differences are statistically significant. For example: "The hybrid approach achieved XX.XX\% accuracy, outperforming pure CNN (XX.XX\%) by X.X percentage points and pure LSTM (XX.XX\%) by X.X percentage points. Statistical testing with paired t-test revealed that the hybrid improvement is significant (p < 0.05)."]}

\subsection{Milestone 1: CNN-Based Approach Results}

\subsubsection{Architecture Comparison}

We evaluated 15+ CNN architectures spanning multiple design families. Table~\ref{tab:cnn_results} presents the complete results.

\begin{table}[h]
\centering
\caption{CNN Architecture Performance Comparison}
\label{tab:cnn_results}
\small
\begin{tabular}{@{}lcccccc@{}}
\toprule
\textbf{Architecture} & \textbf{Params (M)} & \textbf{Acc (\%)} & \textbf{Prec} & \textbf{Rec} & \textbf{F1} & \textbf{Inf. Time (ms)} \\
\midrule
\multicolumn{7}{c}{\textit{Basic CNNs}} \\
\midrule
CNN-1D & \todo{X.X} & \todo{XX.XX} & \todo{X.XXX} & \todo{X.XXX} & \todo{X.XXX} & \todo{XX.X} \\
\midrule
\multicolumn{7}{c}{\textit{ResNet Family}} \\
\midrule
ResNet-18 & \todo{XX.X} & \todo{XX.XX} & \todo{X.XXX} & \todo{X.XXX} & \todo{X.XXX} & \todo{XX.X} \\
ResNet-34 & \todo{XX.X} & \todo{XX.XX} & \todo{X.XXX} & \todo{X.XXX} & \todo{X.XXX} & \todo{XX.X} \\
ResNet-50 & \todo{XX.X} & \todo{XX.XX} & \todo{X.XXX} & \todo{X.XXX} & \todo{X.XXX} & \todo{XX.X} \\
\midrule
\multicolumn{7}{c}{\textit{EfficientNet Family}} \\
\midrule
EfficientNet-B0 & \todo{X.X} & \todo{XX.XX} & \todo{X.XXX} & \todo{X.XXX} & \todo{X.XXX} & \todo{XX.X} \\
EfficientNet-B2 & \todo{X.X} & \todo{XX.XX} & \todo{X.XXX} & \todo{X.XXX} & \todo{X.XXX} & \todo{XX.X} \\
EfficientNet-B4 & \todo{XX.X} & \todo{XX.XX} & \todo{X.XXX} & \todo{X.XXX} & \todo{X.XXX} & \todo{XX.X} \\
\bottomrule
\end{tabular}
\end{table}

\figplaceholder{Figure: Scatter plot showing accuracy vs. parameters for all CNN architectures, highlighting the accuracy-efficiency trade-off}

\figplaceholder{Figure: Confusion matrix for the best-performing CNN architecture (11x11 heatmap)}

\paragraph{Analysis:}

\todo{[This paragraph will analyze: (1) Which CNN family performed best (ResNet vs. EfficientNet vs. basic CNNs), (2) Whether deeper networks outperformed shallower ones, (3) Efficiency considerations (parameters vs. accuracy), (4) Specific architectural features that contributed to performance (skip connections, efficient scaling, etc.). Example: "ResNet-34 achieved the highest accuracy at XX.XX\%, demonstrating that skip connections are beneficial for this task. Interestingly, ResNet-50 showed only marginal improvement (XX.XX\%) despite having significantly more parameters, suggesting potential overfitting on this dataset size. EfficientNet-B2 provided an excellent balance with XX.XX\% accuracy and only X.XM parameters."]}

\subsubsection{Per-Class Performance}

Table~\ref{tab:cnn_per_class} shows per-class metrics for the best CNN model.

\begin{table}[h]
\centering
\caption{Per-Class Performance of Best CNN Model}
\label{tab:cnn_per_class}
\small
\begin{tabular}{@{}lcccc@{}}
\toprule
\textbf{Fault Class} & \textbf{Precision} & \textbf{Recall} & \textbf{F1-Score} & \textbf{Support} \\
\midrule
Healthy & \todo{X.XXX} & \todo{X.XXX} & \todo{X.XXX} & \todo{XX} \\
Misalignment & \todo{X.XXX} & \todo{X.XXX} & \todo{X.XXX} & \todo{XX} \\
Imbalance & \todo{X.XXX} & \todo{X.XXX} & \todo{X.XXX} & \todo{XX} \\
Bearing Clearance & \todo{X.XXX} & \todo{X.XXX} & \todo{X.XXX} & \todo{XX} \\
Lubrication Issue & \todo{X.XXX} & \todo{X.XXX} & \todo{X.XXX} & \todo{XX} \\
Cavitation & \todo{X.XXX} & \todo{X.XXX} & \todo{X.XXX} & \todo{XX} \\
Wear & \todo{X.XXX} & \todo{X.XXX} & \todo{X.XXX} & \todo{XX} \\
Oil Whirl & \todo{X.XXX} & \todo{X.XXX} & \todo{X.XXX} & \todo{XX} \\
Mixed Fault 1 & \todo{X.XXX} & \todo{X.XXX} & \todo{X.XXX} & \todo{XX} \\
Mixed Fault 2 & \todo{X.XXX} & \todo{X.XXX} & \todo{X.XXX} & \todo{XX} \\
Mixed Fault 3 & \todo{X.XXX} & \todo{X.XXX} & \todo{X.XXX} & \todo{XX} \\
\midrule
\textbf{Macro Avg} & \todo{X.XXX} & \todo{X.XXX} & \todo{X.XXX} & \todo{XXX} \\
\textbf{Weighted Avg} & \todo{X.XXX} & \todo{X.XXX} & \todo{X.XXX} & \todo{XXX} \\
\bottomrule
\end{tabular}
\end{table}

\figplaceholder{Figure: Bar chart showing F1-scores for each fault class, identifying strengths and weaknesses}

\paragraph{Observations:}

\todo{[This will discuss: (1) Which fault types are easiest/hardest to classify, (2) Performance on single vs. mixed faults, (3) Common confusion patterns, (4) Potential reasons for class-specific performance differences. Example: "Healthy bearing classification achieved perfect or near-perfect precision (X.XXX) as expected, given its distinct signature. Mixed Fault 2 (bearing clearance + lubrication) proved most challenging with F1-score of X.XXX, likely due to overlapping vibration patterns from multiple degradation mechanisms."]}

\subsubsection{Training Dynamics}

\figplaceholder{Figure: Training and validation curves (loss and accuracy) for best CNN model over 75 epochs}

\figplaceholder{Figure: Learning rate schedule showing cosine annealing decay}

\paragraph{Training Characteristics:}

\todo{[Analysis of: (1) Convergence speed (epochs to reach 95\% of final performance), (2) Overfitting behavior (training vs. validation gap), (3) Effect of early stopping, (4) Stability across different random seeds. Example: "The best CNN model converged rapidly, reaching XX\% validation accuracy within XX epochs. A small train-validation gap of X.X\% suggests limited overfitting. Early stopping triggered at epoch XX, preventing further overfitting observed in preliminary experiments without this regularization."]}

\subsection{Milestone 2: LSTM-Based Approach Results}

\subsubsection{LSTM Variant Comparison}

Table~\ref{tab:lstm_results} compares Vanilla LSTM and BiLSTM configurations.

\begin{table}[h]
\centering
\caption{LSTM Architecture Performance Comparison}
\label{tab:lstm_results}
\begin{tabular}{@{}lcccccc@{}}
\toprule
\textbf{Architecture} & \textbf{Hidden} & \textbf{Params (M)} & \textbf{Acc (\%)} & \textbf{F1} & \textbf{Train Time (h)} & \textbf{Inf. (ms)} \\
\midrule
Vanilla LSTM & 128 & \todo{X.X} & \todo{XX.XX} & \todo{X.XXX} & \todo{X.X} & \todo{XX.X} \\
Vanilla LSTM & 256 & \todo{X.X} & \todo{XX.XX} & \todo{X.XXX} & \todo{X.X} & \todo{XX.X} \\
\midrule
BiLSTM & 128 & \todo{X.X} & \todo{XX.XX} & \todo{X.XXX} & \todo{X.X} & \todo{XX.X} \\
BiLSTM & 256 & \todo{X.X} & \todo{XX.XX} & \todo{X.XXX} & \todo{X.X} & \todo{XX.X} \\
\bottomrule
\end{tabular}
\end{table}

\figplaceholder{Figure: Bar chart comparing Vanilla LSTM vs. BiLSTM performance}

\paragraph{Findings:}

\todo{[Analysis will cover: (1) Performance gap between unidirectional and bidirectional LSTMs, (2) Impact of hidden size on accuracy, (3) Computational trade-offs, (4) Whether bidirectional processing justifies the doubled cost. Example: "BiLSTM with hidden size 256 achieved the best accuracy at XX.XX\%, outperforming Vanilla LSTM by X.X percentage points. This improvement comes at the cost of doubled parameters and ~2× longer inference time. For offline analysis where computational budget permits, BiLSTM is clearly superior."]}

\subsubsection{Comparison with CNN Approach}

Direct comparison between best LSTM and best CNN:

\begin{table}[h]
\centering
\caption{CNN vs. LSTM: Head-to-Head Comparison}
\label{tab:cnn_vs_lstm}
\begin{tabular}{@{}lccccc@{}}
\toprule
\textbf{Metric} & \textbf{Best CNN} & \textbf{Best LSTM} & \textbf{Difference} & \textbf{p-value} & \textbf{Significant?} \\
\midrule
Accuracy (\%) & \todo{XX.XX $\pm$ X.XX} & \todo{XX.XX $\pm$ X.XX} & \todo{$\pm$X.XX} & \todo{X.XXX} & \todo{Yes/No} \\
Macro-F1 & \todo{X.XXX $\pm$ X.XXX} & \todo{X.XXX $\pm$ X.XXX} & \todo{$\pm$X.XXX} & \todo{X.XXX} & \todo{Yes/No} \\
Params (M) & \todo{XX.X} & \todo{X.X} & \todo{$-$XX.X} & - & - \\
Inf. Time (ms) & \todo{XX.X} & \todo{XX.X} & \todo{$\pm$XX.X} & - & - \\
\bottomrule
\end{tabular}
\end{table}

\paragraph{Discussion:}

\todo{[This will interpret: (1) Which approach achieved better accuracy, (2) Magnitude and statistical significance of differences, (3) Advantages and disadvantages of each approach, (4) Scenarios where one might be preferred over the other. Example: "CNN achieved marginally higher accuracy (XX.XX\%) compared to LSTM (XX.XX\%), but the difference of X.X percentage points is not statistically significant (p=X.XXX). However, CNNs demonstrate significantly faster inference (XX.X ms vs. XX.X ms), making them more suitable for real-time applications. The LSTM's advantage in explicit temporal modeling did not translate to substantial performance gains on this dataset, possibly because CNNs capture sufficient temporal information through their receptive fields."]}

\subsubsection{LSTM Attention Analysis}

For LSTM models with attention mechanisms, we visualize learned attention weights:

\figplaceholder{Figure: Attention weight heatmaps for different fault classes, showing which temporal regions the LSTM focuses on for classification}

\paragraph{Interpretability:}

\todo{[Discussion of: (1) Which temporal regions receive high attention for each fault type, (2) Whether attention patterns align with domain knowledge about fault signatures, (3) Differences in attention patterns across fault classes. Example: "Attention visualizations reveal that the LSTM focuses heavily on the first XXX time steps for bearing clearance faults, corresponding to the initial impact in each rotation cycle. In contrast, lubrication faults show more distributed attention across the entire sequence, reflecting their gradual, non-impulsive nature."]}

\subsection{Milestone 3: Hybrid CNN-LSTM Results}

\subsubsection{Recommended Configuration Performance}

Table~\ref{tab:hybrid_recommended} presents results for the three recommended hybrid configurations.

\begin{table}[h]
\centering
\caption{Performance of Recommended Hybrid Configurations}
\label{tab:hybrid_recommended}
\begin{tabular}{@{}lccccc@{}}
\toprule
\textbf{Configuration} & \textbf{Architecture} & \textbf{Params (M)} & \textbf{Acc (\%)} & \textbf{F1} & \textbf{Inf. (ms)} \\
\midrule
Recommended 1 & ResNet34 + BiLSTM & \todo{XX.X} & \todo{XX.XX $\pm$ X.XX} & \todo{X.XXX} & \todo{XX.X} \\
Recommended 2 & EfficientNet-B2 + BiLSTM & \todo{XX.X} & \todo{XX.XX $\pm$ X.XX} & \todo{X.XXX} & \todo{XX.X} \\
Recommended 3 & ResNet18 + LSTM & \todo{XX.X} & \todo{XX.XX $\pm$ X.XX} & \todo{X.XXX} & \todo{XX.X} \\
\bottomrule
\end{tabular}
\end{table}

\figplaceholder{Figure: Comparison of the three recommended configurations across multiple metrics (accuracy, speed, model size)}

\paragraph{Configuration Analysis:}

\todo{[Will discuss: (1) Which configuration achieved best accuracy, (2) Trade-offs between configurations, (3) Which to choose for different deployment scenarios. Example: "Recommended 1 (ResNet34+BiLSTM) achieved the highest accuracy at XX.XX\%, validating its design for maximum performance. Recommended 2 provided competitive accuracy (XX.XX\%) with XX\% fewer parameters, making it ideal for resource-constrained deployment. Recommended 3 offered the fastest inference (XX.X ms) suitable for real-time monitoring, with acceptable accuracy of XX.XX\%."]}

\subsubsection{Ablation Studies}

To understand which design choices matter most, we conducted systematic ablation studies.

\paragraph{Ablation 1: CNN Backbone Selection}

Fixing LSTM configuration (BiLSTM, hidden=256, 2 layers, mean pooling), we varied the CNN backbone:

\begin{table}[h]
\centering
\caption{Ablation Study: CNN Backbone Selection}
\label{tab:ablation_cnn}
\begin{tabular}{@{}lcccc@{}}
\toprule
\textbf{CNN Backbone} & \textbf{CNN Params (M)} & \textbf{Total Params (M)} & \textbf{Accuracy (\%)} & \textbf{$\Delta$ from Best} \\
\midrule
CNN-1D & \todo{X.X} & \todo{X.X} & \todo{XX.XX} & \todo{-X.XX} \\
ResNet-18 & \todo{XX.X} & \todo{XX.X} & \todo{XX.XX} & \todo{-X.XX} \\
ResNet-34 & \todo{XX.X} & \todo{XX.X} & \todo{XX.XX} & \todo{0.00} \\
ResNet-50 & \todo{XX.X} & \todo{XX.X} & \todo{XX.XX} & \todo{-X.XX} \\
EfficientNet-B0 & \todo{X.X} & \todo{X.X} & \todo{XX.XX} & \todo{-X.XX} \\
EfficientNet-B2 & \todo{X.X} & \todo{X.X} & \todo{XX.XX} & \todo{-X.XX} \\
EfficientNet-B4 & \todo{XX.X} & \todo{XX.X} & \todo{XX.XX} & \todo{-X.XX} \\
\bottomrule
\end{tabular}
\end{table}

\figplaceholder{Figure: Line plot showing accuracy vs. CNN backbone complexity}

\todo{[Discussion will explain which CNN backbones work best with LSTM, whether deeper CNNs provide better features, and the accuracy-efficiency trade-off. Example: "ResNet-34 emerged as the optimal CNN backbone with XX.XX\% accuracy. Surprisingly, the deeper ResNet-50 underperformed (XX.XX\%), suggesting that its higher-capacity features may be over-specialized for end-to-end classification. EfficientNet-B2 provided strong performance (XX.XX\%) with excellent parameter efficiency."]}

\paragraph{Ablation 2: LSTM Type}

Fixing CNN backbone (ResNet-34), we compared LSTM types:

\begin{table}[h]
\centering
\caption{Ablation Study: LSTM Type Selection}
\label{tab:ablation_lstm_type}
\begin{tabular}{@{}lcccc@{}}
\toprule
\textbf{LSTM Type} & \textbf{Hidden Size} & \textbf{Params (M)} & \textbf{Accuracy (\%)} & \textbf{Inference (ms)} \\
\midrule
Vanilla LSTM & 128 & \todo{XX.X} & \todo{XX.XX} & \todo{XX.X} \\
Vanilla LSTM & 256 & \todo{XX.X} & \todo{XX.XX} & \todo{XX.X} \\
BiLSTM & 128 & \todo{XX.X} & \todo{XX.XX} & \todo{XX.X} \\
BiLSTM & 256 & \todo{XX.X} & \todo{XX.XX} & \todo{XX.X} \\
\bottomrule
\end{tabular}
\end{table}

\todo{[Will discuss the accuracy gain from bidirectional processing and larger hidden sizes, and whether the computational cost is justified.]}

\paragraph{Ablation 3: Temporal Pooling Strategy}

Fixing architecture (ResNet-34 + BiLSTM-256), we compared pooling methods:

\begin{table}[h]
\centering
\caption{Ablation Study: Temporal Pooling Methods}
\label{tab:ablation_pooling}
\begin{tabular}{@{}lccc@{}}
\toprule
\textbf{Pooling Method} & \textbf{Additional Params} & \textbf{Accuracy (\%)} & \textbf{$\Delta$ from Mean} \\
\midrule
Mean & 0 & \todo{XX.XX} & \todo{0.00} \\
Max & 0 & \todo{XX.XX} & \todo{$\pm$X.XX} \\
Last Timestep & 0 & \todo{XX.XX} & \todo{$\pm$X.XX} \\
Attention & \todo{XXK} & \todo{XX.XX} & \todo{$\pm$X.XX} \\
\bottomrule
\end{tabular}
\end{table}

\figplaceholder{Figure: Visualization of attention weights learned by the attention pooling mechanism}

\todo{[Discussion of which pooling method works best and whether attention's added complexity yields meaningful gains. Example: "Mean pooling achieved XX.XX\% accuracy, matching or exceeding other strategies. Attention pooling marginally improved performance (XX.XX\%), but the X.XX\% gain may not justify the added complexity for most applications. Interestingly, last timestep pooling significantly underperformed (XX.XX\%), suggesting that compressing all information into the final LSTM state is suboptimal for this task."]}

\paragraph{Ablation 4: CNN Freezing (Transfer Learning)}

Comparing frozen CNN (weights fixed) vs. end-to-end fine-tuning:

\begin{table}[h]
\centering
\caption{Ablation Study: CNN Freezing Strategy}
\label{tab:ablation_freezing}
\begin{tabular}{@{}lccc@{}}
\toprule
\textbf{Training Strategy} & \textbf{Accuracy (\%)} & \textbf{Training Time (h)} & \textbf{GPU Memory (GB)} \\
\midrule
Frozen CNN & \todo{XX.XX} & \todo{X.X} & \todo{X.X} \\
Fine-tuned CNN & \todo{XX.XX} & \todo{X.X} & \todo{X.X} \\
\midrule
Improvement & \todo{$+$X.XX} & \todo{$-$X.X} & \todo{$+$X.X} \\
\bottomrule
\end{tabular}
\end{table}

\todo{[Will analyze the accuracy-speed trade-off of transfer learning. Example: "End-to-end fine-tuning improved accuracy by X.XX percentage points (XX.XX\% vs. XX.XX\%), demonstrating that task-specific CNN adaptation is beneficial. However, this came at the cost of X.X× longer training time. For rapid prototyping or limited computational budgets, frozen CNN provides a reasonable compromise."]}

\subsubsection{Hybrid vs. Individual Approaches}

Comprehensive comparison of the best model from each milestone:

\begin{table}[h]
\centering
\caption{Comprehensive Three-Way Comparison}
\label{tab:three_way}
\begin{tabular}{@{}lcccccc@{}}
\toprule
\textbf{Approach} & \textbf{Model} & \textbf{Acc (\%)} & \textbf{F1} & \textbf{Params (M)} & \textbf{Size (MB)} & \textbf{Inf. (ms)} \\
\midrule
CNN & \todo{ResNet-XX} & \todo{XX.XX} & \todo{X.XXX} & \todo{XX.X} & \todo{XXX} & \todo{XX.X} \\
LSTM & \todo{BiLSTM-XXX} & \todo{XX.XX} & \todo{X.XXX} & \todo{X.X} & \todo{XX} & \todo{XX.X} \\
Hybrid & \todo{ResNetXX+BiLSTM} & \todo{XX.XX} & \todo{X.XXX} & \todo{XX.X} & \todo{XXX} & \todo{XX.X} \\
\bottomrule
\end{tabular}
\end{table}

\figplaceholder{Figure: Radar chart comparing the three approaches across multiple dimensions: accuracy, speed, model size, training time, robustness}

\paragraph{Comparative Analysis:}

\todo{[Comprehensive discussion addressing: (1) Did the hybrid approach achieve the best accuracy? (2) What is the performance gain over individual approaches? (3) Is the gain worth the added complexity? (4) When would you choose each approach? Example detailed analysis: "The hybrid approach achieved XX.XX\% accuracy, surpassing CNN (XX.XX\%) by X.X points and LSTM (XX.XX\%) by X.X points. Statistical testing confirms significance (p<0.05 for both comparisons). This validates our hypothesis that combining spatial and temporal modeling yields superior performance. However, the hybrid model is significantly larger (XX.XM parameters vs. XX.XM for CNN and X.XM for LSTM) and slower (XX.X ms vs. XX.X ms for CNN). For deployment scenarios prioritizing accuracy over computational constraints, the hybrid approach is clearly superior. For real-time edge deployment, pure CNN offers the best accuracy-efficiency trade-off."]}

\subsection{Robustness Analysis}

\subsubsection{Performance Under Noise}

We evaluated all three approaches under varying noise levels:

\begin{table}[h]
\centering
\caption{Accuracy Degradation Under Additive Noise}
\label{tab:noise_robustness}
\begin{tabular}{@{}lcccc@{}}
\toprule
\textbf{Approach} & \textbf{Clean} & \textbf{$\sigma$=0.01} & \textbf{$\sigma$=0.05} & \textbf{$\sigma$=0.1} \\
\midrule
CNN & \todo{XX.XX\%} & \todo{XX.XX\%} & \todo{XX.XX\%} & \todo{XX.XX\%} \\
LSTM & \todo{XX.XX\%} & \todo{XX.XX\%} & \todo{XX.XX\%} & \todo{XX.XX\%} \\
Hybrid & \todo{XX.XX\%} & \todo{XX.XX\%} & \todo{XX.XX\%} & \todo{XX.XX\%} \\
\bottomrule
\end{tabular}
\end{table}

\figplaceholder{Figure: Line plot showing accuracy vs. noise level for all three approaches}

\paragraph{Robustness Findings:}

\todo{[Will discuss: (1) Which approach is most robust to noise, (2) Rate of performance degradation, (3) Potential reasons for different robustness levels. Example: "All approaches demonstrated graceful degradation with increasing noise. At moderate noise ($\sigma$=0.05), accuracy dropped by X.X\%, X.X\%, and X.X\% for CNN, LSTM, and Hybrid respectively. The hybrid approach showed superior robustness, likely due to its hierarchical processing where CNN features provide some noise filtering before LSTM processing."]}

\subsubsection{Confusion Analysis}

Detailed analysis of common misclassification patterns:

\figplaceholder{Figure: Confusion matrices for CNN, LSTM, and Hybrid approaches side-by-side for comparison}

\paragraph{Misclassification Patterns:}

\todo{[Analysis of: (1) Which fault pairs are most frequently confused, (2) Whether confusion patterns differ across approaches, (3) Physical interpretation of confusions. Example: "All approaches occasionally confused Mixed Fault 2 (bearing clearance + lubrication) with Lubrication Issue alone, achieving only XX-XX\% precision for this class. This is physically reasonable as lubrication faults likely dominate the vibration signature in the mixed fault. Interestingly, the hybrid approach showed fewer confusions between Misalignment and Imbalance (X% error rate vs. X% for CNN), suggesting its temporal modeling better distinguishes these faults with different time-domain characteristics."]}

\subsection{Computational Efficiency Analysis}

\subsubsection{Training Efficiency}

\begin{table}[h]
\centering
\caption{Training Time and Resource Requirements}
\label{tab:training_efficiency}
\begin{tabular}{@{}lccccc@{}}
\toprule
\textbf{Approach} & \textbf{Time/Epoch (s)} & \textbf{Total Time (h)} & \textbf{GPU Mem (GB)} & \textbf{Convergence Epoch} \\
\midrule
CNN (ResNet-34) & \todo{XXX} & \todo{X.X} & \todo{X.X} & \todo{XX} \\
LSTM (BiLSTM-256) & \todo{XXX} & \todo{X.X} & \todo{X.X} & \todo{XX} \\
Hybrid (ResNet34+BiLSTM) & \todo{XXX} & \todo{X.X} & \todo{X.X} & \todo{XX} \\
\bottomrule
\end{tabular}
\end{table}

\paragraph{Training Characteristics:}

\todo{[Discussion of training efficiency, convergence speed, and resource requirements across approaches.]}

\subsubsection{Inference Efficiency}

\begin{table}[h]
\centering
\caption{Inference Performance on Different Hardware}
\label{tab:inference_efficiency}
\begin{tabular}{@{}lcccc@{}}
\toprule
\textbf{Approach} & \textbf{GPU (ms/sample)} & \textbf{CPU (ms/sample)} & \textbf{Throughput (samples/s)} & \textbf{Model Size (MB)} \\
\midrule
CNN & \todo{XX.X} & \todo{XXX} & \todo{XXX} & \todo{XXX} \\
LSTM & \todo{XX.X} & \todo{XXX} & \todo{XXX} & \todo{XX} \\
Hybrid & \todo{XX.X} & \todo{XXX} & \todo{XXX} & \todo{XXX} \\
\bottomrule
\end{tabular}
\end{table}

\paragraph{Deployment Considerations:}

\todo{[Will discuss: (1) Real-time capability of each approach, (2) Suitability for edge vs. cloud deployment, (3) Recommendations based on deployment constraints. Example: "For real-time monitoring requiring sub-100ms latency, CNN is the only viable option (XX.X ms inference time). LSTM and hybrid approaches exceed this threshold (XX.X ms and XX.X ms respectively) but are suitable for offline batch analysis. On CPU hardware, all approaches become significantly slower, with hybrid requiring XXX ms per sample."]}

\subsection{Summary of Key Results}

\begin{enumerate}
    \item \textbf{Overall Performance:} \todo{[Hybrid/CNN/LSTM] achieved the highest test accuracy of XX.XX\%, establishing new state-of-the-art on this dataset.}

    \item \textbf{CNN Findings:} \todo{[ResNet-XX/EfficientNet-XX] provided the best accuracy-efficiency trade-off among CNN architectures with XX.XX\% accuracy and XX.XM parameters.}

    \item \textbf{LSTM Findings:} \todo{BiLSTM outperformed vanilla LSTM by X.X percentage points, justifying bidirectional processing for offline analysis.}

    \item \textbf{Hybrid Advantages:} \todo{The hybrid approach improved accuracy by X.X points over pure CNN and X.X points over pure LSTM, demonstrating the benefit of combining spatial and temporal modeling.}

    \item \textbf{Ablation Insights:} \todo{CNN backbone choice proved most critical (X.X\% accuracy range), followed by LSTM type (X.X\% difference) and pooling method (X.X\% difference).}

    \item \textbf{Robustness:} \todo{All approaches maintained >XX\% accuracy under moderate noise ($\sigma$=0.05), with hybrid showing best robustness.}

    \item \textbf{Efficiency:} \todo{CNN offered fastest inference (XX.X ms), while hybrid sacrificed speed for accuracy. LSTM provided a middle ground.}
\end{enumerate}

These results comprehensively validate our three-milestone approach and demonstrate that each methodology offers distinct advantages depending on application requirements.


\section{Conclusion and Future Work}
\label{sec:conclusion}
This study presented a comprehensive investigation of deep learning approaches for automated bearing fault diagnosis, systematically evaluating CNN-based spatial feature learning, LSTM-based temporal sequence modeling, and novel configurable hybrid architectures.

\subsection{Principal Findings}

Our research yielded several key findings that advance the field of intelligent fault diagnosis:

\subsubsection{Comparative Performance}

\todo{[After completing experiments, this will summarize: (1) Final ranking of the three approaches, (2) Magnitudes of performance differences, (3) Statistical significance of results. Example: "The hybrid CNN-LSTM approach achieved the highest test accuracy of XX.XX\%, outperforming pure CNN (XX.XX\%) by X.X percentage points and pure LSTM (XX.XX\%) by X.X points. Statistical testing confirmed that these improvements are significant (p<0.05), validating our hypothesis that combining spatial and temporal modeling yields superior diagnostic performance."]}

\subsubsection{Architectural Insights}

Through systematic evaluation of 15+ CNN architectures, 2 LSTM variants, and 56+ hybrid configurations, we identified several architectural principles:

\begin{enumerate}
    \item \textbf{CNN Depth vs. Performance:} \todo{[Will state whether deeper CNNs consistently outperformed shallower ones, or if there was a point of diminishing returns. Example: "ResNet-34 provided optimal accuracy-efficiency balance. Deeper variants (ResNet-50) showed only marginal gains, suggesting that excessive depth may not be beneficial for this dataset size."]}

    \item \textbf{Efficient Architectures:} \todo{[Findings about EfficientNet performance. Example: "EfficientNet-B2 achieved competitive performance (XX.XX\%) with significantly fewer parameters than ResNet counterparts, demonstrating the value of compound scaling for parameter-efficient models."]}

    \item \textbf{Bidirectional Processing Value:} \todo{[LSTM findings. Example: "Bidirectional LSTMs consistently outperformed unidirectional variants by X-X percentage points across all configurations, justifying the doubled computational cost for offline analysis scenarios."]}

    \item \textbf{Hybrid Integration Strategies:} \todo{[Hybrid findings. Example: "Mean temporal pooling proved most effective for aggregating LSTM outputs, outperforming max pooling and last-timestep strategies. Attention mechanisms provided marginal improvements (X.XX\%) at significant complexity cost."]}

    \item \textbf{Transfer Learning Efficacy:} \todo{[CNN freezing results. Example: "End-to-end fine-tuning of CNN backbones improved hybrid performance by X.XX percentage points over frozen CNN features, demonstrating that task-specific adaptation is beneficial despite increased training time."]}
\end{enumerate}

\subsubsection{Configurable Framework Contribution}

Our configurable hybrid framework represents a significant methodological contribution. Unlike prior work that coupled specific CNN and LSTM architectures, our modular design enabled:

\begin{itemize}
    \item Systematic exploration of 56+ architectural combinations through simple parameter changes
    \item Identification of optimal CNN-LSTM pairings for different deployment scenarios
    \item Application-specific optimization based on accuracy, efficiency, or latency constraints
    \item Rapid prototyping of new configurations without architectural reimplementation
\end{itemize}

This flexibility proved valuable in our ablation studies and will facilitate future research exploring alternative backbone architectures (e.g., Transformers, Capsule Networks).

\subsubsection{Practical Deployment Insights}

Beyond academic performance metrics, our study provides actionable insights for industrial deployment:

\begin{enumerate}
    \item \textbf{Real-Time Monitoring:} \todo{CNN models offer sub-XX-ms inference times suitable for real-time condition monitoring applications.}

    \item \textbf{Offline Analysis:} \todo{Hybrid approaches provide highest accuracy (XX.XX\%) for batch processing and detailed fault analysis.}

    \item \textbf{Edge Deployment:} \todo{EfficientNet-B2 (pure CNN or hybrid) provides optimal balance for resource-constrained edge devices with X.XM parameters and XX.X ms inference.}

    \item \textbf{Cloud Deployment:} \todo{ResNet-50 hybrid configurations can be deployed on cloud infrastructure where computational resources are abundant and maximum accuracy is prioritized.}
\end{enumerate}

\subsection{Limitations and Considerations}

While our study provides comprehensive insights, several limitations should be acknowledged:

\subsubsection{Dataset Constraints}

\begin{itemize}
    \item \textbf{Single Bearing Type:} All data derived from SKF 6205-2RS bearings. Generalization to different bearing geometries, sizes, and manufacturers requires validation.

    \item \textbf{Laboratory Conditions:} Data collected under controlled conditions may not fully represent industrial environments with multiple concurrent noise sources, temperature variations, and transient operating conditions.

    \item \textbf{Synthetic Faults:} EDM-introduced defects, while reproducible, may differ from naturally occurring degradation patterns in long-term industrial operation.

    \item \textbf{Limited Mixed Faults:} Only three mixed fault categories included. Real-world machinery often exhibits more complex combinations of multiple simultaneous degradation mechanisms.
\end{itemize}

\subsubsection{Methodological Limitations}

\begin{itemize}
    \item \textbf{Fixed Operating Conditions:} Most experiments conducted at constant speed and load. Variable speed and load conditions present additional challenges for fault diagnosis.

    \item \textbf{Single Sensor Modality:} Only vibration data utilized. Industrial systems often employ multiple sensor types (vibration, temperature, acoustic emission, current) that could be fused for improved diagnosis.

    \item \textbf{Computational Environment:} All experiments performed on high-end GPU hardware. Performance on resource-constrained edge devices requires further investigation.
\end{itemize}

\subsubsection{Scope Limitations}

\begin{itemize}
    \item \textbf{Classification Only:} Study focused on fault classification. Early fault detection, severity estimation, and remaining useful life prediction are complementary problems not addressed.

    \item \textbf{Interpretability:} While we explored attention mechanisms, deeper investigation of model interpretability through saliency maps, activation visualizations, and adversarial analysis could enhance trust and adoption.

    \item \textbf{Uncertainty Quantification:} Models provide point predictions without uncertainty estimates. Bayesian approaches or ensemble methods could quantify prediction confidence.
\end{itemize}

\subsection{Future Research Directions}

This work opens several promising avenues for future research:

\subsubsection{Architectural Extensions}

\begin{enumerate}
    \item \textbf{Transformer-Based Models:} Self-attention mechanisms in Transformers have shown remarkable success in sequence modeling. Investigating Transformer backbones in place of LSTMs could capture long-range dependencies more effectively.

    \item \textbf{Multi-Scale Architectures:} Developing architectures that explicitly process signals at multiple temporal scales (e.g., wavelet-based multi-resolution analysis) could better capture fault signatures at different frequency bands.

    \item \textbf{Capsule Networks:} Capsule networks' ability to preserve spatial hierarchies and relationships could offer advantages for capturing complex fault patterns.

    \item \textbf{Graph Neural Networks:} For systems with multiple connected components, GNNs could model spatial relationships between sensors and propagation of fault effects.
\end{enumerate}

\subsubsection{Data and Generalization}

\begin{enumerate}
    \item \textbf{Cross-Dataset Evaluation:} Validating models on additional bearing datasets (e.g., Paderborn University, XJTU-SY, IMS) would establish generalization capabilities.

    \item \textbf{Domain Adaptation:} Investigating transfer learning and domain adaptation techniques to apply models trained on one bearing type to others would enhance practical utility.

    \item \textbf{Few-Shot Learning:} Developing approaches that can learn new fault types from limited examples would address the challenge of rare fault data collection.

    \item \textbf{Multi-Sensor Fusion:} Integrating vibration, temperature, acoustic, and current signatures through multi-modal deep learning could improve robustness and accuracy.
\end{enumerate}

\subsubsection{Advanced Methodologies}

\begin{enumerate}
    \item \textbf{Uncertainty Quantification:} Implementing Bayesian neural networks, Monte Carlo dropout, or deep ensembles to provide confidence intervals on predictions.

    \item \textbf{Explainable AI:} Developing interpretability techniques (SHAP, LIME, attention visualization, saliency maps) to build trust and enable root cause analysis.

    \item \textbf{Online Learning:} Creating models that can adapt to evolving machinery conditions through continual learning without catastrophic forgetting.

    \item \textbf{Prognostics:} Extending diagnostic models to predict remaining useful life and fault progression trajectories.

    \item \textbf{Active Learning:} Developing strategies to intelligently select informative samples for labeling, reducing annotation costs.
\end{enumerate}

\subsubsection{Deployment and Implementation}

\begin{enumerate}
    \item \textbf{Model Compression:} Investigating quantization, pruning, and knowledge distillation to reduce model size for edge deployment while maintaining accuracy.

    \item \textbf{Real-Time Implementation:} Optimizing models for real-time inference on embedded systems and field-programmable gate arrays (FPGAs).

    \item \textbf{Federated Learning:} Developing distributed training frameworks that preserve data privacy while leveraging data from multiple industrial sites.

    \item \textbf{Industrial Validation:} Conducting pilot deployments in operational industrial facilities to assess performance under realistic conditions.
\end{enumerate}

\subsubsection{Broader Applications}

\begin{enumerate}
    \item \textbf{Other Rotating Machinery:} Applying methodologies to gearboxes, pumps, compressors, and turbines to establish broader applicability.

    \item \textbf{Structural Health Monitoring:} Adapting approaches for civil infrastructure monitoring (bridges, buildings, pipelines).

    \item \textbf{Medical Diagnosis:} Transferring techniques to physiological signal analysis (ECG, EEG) for medical fault detection.

    \item \textbf{Process Monitoring:} Extending to chemical processes, power systems, and other continuous monitoring applications.
\end{enumerate}

\subsection{Implications for Practice}

This research has direct implications for industrial predictive maintenance:

\subsubsection{Technology Readiness}

The implemented models achieve \todo{XX.XX\%} accuracy on a benchmark dataset, demonstrating readiness for pilot deployment. The three-tier approach (CNN for real-time, LSTM for detailed analysis, hybrid for maximum accuracy) provides flexible options matching different operational requirements.

\subsubsection{Implementation Recommendations}

Based on our findings, we recommend:

\begin{enumerate}
    \item \textbf{Start with CNN Baselines:} ResNet-34 provides excellent performance with reasonable computational cost, suitable for initial deployment.

    \item \textbf{Consider Hybrid for Critical Assets:} For high-value machinery where downtime is extremely costly, the hybrid approach's superior accuracy (\todo{XX.XX\%}) justifies additional computational resources.

    \item \textbf{Optimize for Deployment Target:} Use our configurable framework to select architectures matching available hardware (edge device vs. cloud server).

    \item \textbf{Establish Continuous Monitoring:} Deploy models in shadow mode initially, comparing predictions with expert judgments to build confidence before full automation.

    \item \textbf{Plan for Model Updates:} Collect and label new fault data regularly, retraining models to adapt to evolving machinery conditions.
\end{enumerate}

\subsubsection{Economic Benefits}

Accurate fault diagnosis enables:
\begin{itemize}
    \item \textbf{Reduced Downtime:} Early detection prevents catastrophic failures, reducing unplanned downtime by 30-50\% (industry estimates).
    \item \textbf{Optimized Maintenance:} Transitioning from time-based to condition-based maintenance reduces unnecessary interventions by 20-30\%.
    \item \textbf{Extended Equipment Life:} Early intervention prevents secondary damage, extending bearing life by 15-25\%.
    \item \textbf{Safety Improvements:} Preventing unexpected failures reduces safety incidents in industrial environments.
\end{itemize}

For a typical industrial facility, these benefits can translate to millions of dollars in annual savings, far exceeding the implementation cost of automated diagnostic systems.

\subsection{Concluding Remarks}

This dissertation presented a systematic investigation of deep learning for bearing fault diagnosis, progressing from CNNs (spatial) to LSTMs (temporal) to hybrid architectures (integrated). Each milestone delivered production-ready implementations with comprehensive documentation, enabling both research advancement and practical deployment.

Our configurable hybrid framework represents a significant contribution, enabling flexible exploration of architecture combinations and application-specific optimization. The framework's modularity facilitates future extensions with new CNN backbones, LSTM variants, or entirely different components (e.g., Transformers).

The comprehensive experimental results, spanning \todo{XX+} architectural configurations and rigorous ablation studies, provide evidence-based guidance for architecture selection. We demonstrated that \todo{[highest-performing approach]} achieves \todo{XX.XX\%} accuracy, advancing state-of-the-art on the CWRU benchmark.

Beyond academic contributions, this work delivers practical value through open-source implementations, deployment-ready models, and clear recommendations for industrial adoption. The code, trained models, and documentation are publicly available, facilitating reproducibility and enabling practitioners to leverage our findings.

Intelligent fault diagnosis represents a critical component of Industry 4.0 and smart manufacturing initiatives. As industrial systems become increasingly instrumented and interconnected, the need for automated, accurate, and scalable diagnostic solutions will only grow. This research contributes to that vision by demonstrating that deep learning can achieve expert-level diagnostic performance while offering flexibility, scalability, and deployment versatility.

The journey from raw vibration signals to accurate fault classification, traversing CNNs, LSTMs, and hybrid architectures, demonstrates the power of deep learning for industrial applications. While challenges remain---particularly in generalization across different machinery types and operating conditions---the foundation established in this work provides a solid platform for continued progress toward fully automated, intelligent condition monitoring systems that enhance safety, reliability, and efficiency in industrial operations worldwide.


% ============================================================================
% REFERENCES
% ============================================================================

\clearpage
\bibliographystyle{IEEEtran}
\bibliography{references}

% ============================================================================
% APPENDICES (Optional)
% ============================================================================

\clearpage
\appendix

\section{Model Architectures}
\label{app:architectures}

This appendix provides detailed architectural diagrams and parameter counts for all implemented models.

\subsection{CNN Architectures}

\figplaceholder{Detailed CNN architecture diagrams (ResNet18, ResNet34, ResNet50, EfficientNet variants)}

\subsection{LSTM Architectures}

\figplaceholder{Detailed LSTM architecture diagrams (Vanilla LSTM, BiLSTM)}

\subsection{Hybrid Architectures}

\figplaceholder{Detailed hybrid architecture diagrams showing CNN-LSTM integration}

\section{Training Curves}
\label{app:training_curves}

\figplaceholder{Training and validation curves for all models across all milestones}

\section{Hyperparameter Configurations}
\label{app:hyperparameters}

\tabplaceholder{Complete hyperparameter tables for all experiments}

\end{document}
