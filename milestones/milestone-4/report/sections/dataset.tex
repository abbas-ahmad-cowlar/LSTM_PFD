This study utilizes the Case Western Reserve University (CWRU) Bearing Data Center dataset, one of the most widely cited benchmark datasets in bearing fault diagnosis research. This section provides comprehensive details about the dataset characteristics, fault types, data collection methodology, and preprocessing procedures.

\subsection{Experimental Test Rig}

The CWRU dataset was collected from a test rig consisting of a 2-horsepower (hp) motor driving a shaft through a torque transducer/encoder to a dynamometer. Vibration data were measured using accelerometers attached to both the motor housing and the drive end bearing housing. The test rig allowed precise control of motor speed and load conditions, enabling systematic data collection across various operating scenarios.

Accelerometers were positioned at the 12 o'clock position on both the drive end and fan end of the motor housing. The sensors used were industrial-grade accelerometers with high sensitivity and wide frequency response, ensuring accurate capture of bearing vibration characteristics. Data acquisition was performed using a 16-channel DAT recorder, providing high-fidelity signal recording.

\subsection{Bearing Specifications}

The test bearings used in the experiments are SKF deep-groove ball bearings. Table~\ref{tab:bearing_specs} presents the key specifications of the test bearings used in data collection.

\begin{table}[h]
\centering
\caption{Bearing Specifications}
\label{tab:bearing_specs}
\begin{tabular}{@{}ll@{}}
\toprule
\textbf{Parameter} & \textbf{Value} \\
\midrule
Bearing Type & SKF 6205-2RS JEM \\
Number of Rolling Elements & 9 \\
Ball Diameter & 7.94 mm \\
Pitch Diameter & 39.04 mm \\
Contact Angle & 0° \\
\bottomrule
\end{tabular}
\end{table}

These specifications enable calculation of theoretical bearing fault frequencies, which are critical for validating diagnostic results and understanding the physical mechanisms underlying observed vibration patterns.

\subsection{Fault Introduction Methodology}

Faults were introduced into the test bearings using electro-discharge machining (EDM). This controlled fault seeding method ensures reproducible defects with precise dimensions. Three types of single-point defects were created:

\begin{itemize}
    \item \textbf{Inner Race Faults:} Defects machined on the inner race surface
    \item \textbf{Outer Race Faults:} Defects positioned at the 6 o'clock load zone on the outer race
    \item \textbf{Ball Faults:} Defects introduced on the rolling elements
\end{itemize}

For this study, we utilize an extended version of the CWRU dataset that includes additional fault categories beyond the standard inner race, outer race, and ball defects. Our dataset encompasses 11 distinct fault classes representing realistic industrial bearing failure modes.

\subsection{Fault Categories}

Table~\ref{tab:fault_categories} summarizes the 11 fault categories included in our dataset.

\begin{table}[h]
\centering
\caption{Bearing Fault Categories}
\label{tab:fault_categories}
\begin{tabular}{@{}clp{7cm}@{}}
\toprule
\textbf{Class} & \textbf{Fault Type} & \textbf{Description} \\
\midrule
0 & Healthy & Normal bearing operation without defects \\
1 & Misalignment & Shaft or bearing misalignment causing non-uniform load distribution \\
2 & Imbalance & Rotor imbalance leading to synchronous vibration \\
3 & Bearing Clearance & Excessive internal clearance due to wear or manufacturing tolerance \\
4 & Lubrication Issue & Insufficient or contaminated lubrication \\
5 & Cavitation & Fluid cavitation causing surface pitting \\
6 & Wear & Progressive surface degradation through abrasion \\
7 & Oil Whirl & Instability phenomenon in fluid film bearings \\
8 & Mixed Fault 1 & Combination of misalignment and imbalance \\
9 & Mixed Fault 2 & Combination of bearing clearance and lubrication issues \\
10 & Mixed Fault 3 & Combination of wear and cavitation \\
\bottomrule
\end{tabular}
\end{table}

The inclusion of mixed fault categories reflects realistic industrial conditions where multiple degradation mechanisms often occur simultaneously. This multi-class classification problem presents significant challenges for diagnostic algorithms.

\subsection{Data Acquisition Parameters}

Vibration signals were sampled at multiple rates depending on the motor speed:

\begin{itemize}
    \item 12 kHz sampling rate for motor speeds of 1797 rpm
    \item 48 kHz sampling rate for motor speeds of 1772 rpm and 1750 rpm
\end{itemize}

For consistency across all experiments in this study, we standardize the data processing to a uniform sampling rate of \textbf{20.48 kHz}. This rate provides adequate frequency resolution to capture bearing fault signatures while maintaining computational efficiency.

\subsection{Dataset Composition}

Our complete dataset comprises:

\begin{itemize}
    \item \textbf{Total Samples:} 1,430 vibration signal segments
    \item \textbf{Signal Length:} 102,400 samples per segment
    \item \textbf{Duration:} Approximately 5 seconds per segment at 20.48 kHz sampling rate
    \item \textbf{Number of Classes:} 11 fault categories (including healthy condition)
    \item \textbf{Class Distribution:} Approximately 130 samples per fault class
\end{itemize}

The dataset exhibits balanced class distribution, which is advantageous for training deep learning models as it prevents class imbalance issues that could bias the classifier toward majority classes.

\subsection{Data Format}

All vibration data are stored in MATLAB .mat file format, with each file containing:

\begin{itemize}
    \item Raw vibration signal (1D array of acceleration values)
    \item Sampling rate information
    \item Fault type label
    \item Operating condition metadata (motor speed, load)
\end{itemize}

This structured format facilitates reproducible data loading and preprocessing across all experimental implementations.

\subsection{Data Preprocessing}

We apply minimal preprocessing to preserve the raw signal characteristics and allow deep learning models to learn appropriate representations directly from data. The preprocessing pipeline consists of:

\begin{enumerate}
    \item \textbf{Normalization:} Each signal is normalized to zero mean and unit variance:
    \begin{equation}
    x_{norm} = \frac{x - \mu}{\sigma}
    \end{equation}
    where $\mu$ and $\sigma$ are the signal mean and standard deviation, respectively.

    \item \textbf{Segmentation:} Long continuous recordings are segmented into fixed-length windows of 102,400 samples. This length captures multiple rotation cycles while maintaining manageable computational requirements.

    \item \textbf{Quality Control:} Signals are inspected for anomalies such as clipping, excessive noise, or sensor failures. Any corrupted segments are excluded from the dataset.
\end{enumerate}

No additional filtering, feature extraction, or signal transformation is applied, ensuring that deep learning models receive raw vibration data and must learn all relevant features autonomously.

\subsection{Data Splitting Strategy}

To ensure unbiased evaluation and fair comparison across all approaches, we adopt a stratified random split:

\begin{itemize}
    \item \textbf{Training Set:} 60\% of data (approximately 858 samples)
    \item \textbf{Validation Set:} 20\% of data (approximately 286 samples)
    \item \textbf{Test Set:} 20\% of data (approximately 286 samples)
\end{itemize}

Stratification ensures that all fault classes are proportionally represented in each subset, preventing evaluation bias. The same data split is used consistently across all three milestones (CNN, LSTM, and Hybrid approaches) to enable direct performance comparison.

The validation set is used for hyperparameter tuning and model selection during development, while the test set remains held-out for final performance evaluation. This protocol prevents overfitting to the test set and provides realistic estimates of generalization performance.

\subsection{Data Augmentation}

To improve model generalization and robustness, we employ several data augmentation techniques during training:

\begin{enumerate}
    \item \textbf{Additive Gaussian Noise:} Random noise is added to simulate sensor noise and environmental interference:
    \begin{equation}
    x_{aug} = x + \mathcal{N}(0, \alpha^2)
    \end{equation}
    where $\alpha$ controls the noise level (typically 0.01 to 0.05).

    \item \textbf{Amplitude Scaling:} Signals are randomly scaled to simulate varying sensor gains and signal strengths:
    \begin{equation}
    x_{aug} = \beta \cdot x
    \end{equation}
    where $\beta \sim \mathcal{U}(0.8, 1.2)$ is uniformly sampled.

    \item \textbf{Time Shifting:} Circular shifting simulates different starting positions within the rotation cycle:
    \begin{equation}
    x_{aug} = \text{circshift}(x, k)
    \end{equation}
    where $k$ is randomly sampled from $[0, 1024]$.
\end{enumerate}

These augmentation strategies are applied probabilistically during training (with 50\% probability per augmentation) to increase dataset diversity without fundamentally altering fault signatures. Augmentation is not applied to validation or test sets to ensure fair evaluation.

\subsection{Dataset Characteristics and Challenges}

While the CWRU dataset is widely used and well-understood, it presents several challenges:

\begin{itemize}
    \item \textbf{Controlled Environment:} Data collected under laboratory conditions may not fully represent the complexity of industrial environments with multiple noise sources and varying operating conditions.

    \item \textbf{Single Bearing Type:} All data come from the same bearing model, which may limit generalization to other bearing geometries and manufacturers.

    \item \textbf{Synthetic Faults:} EDM-introduced faults, while reproducible, may differ from naturally occurring degradation in real-world applications.

    \item \textbf{Limited Mixed Faults:} While we include three mixed fault categories, real-world machinery often exhibits even more complex failure mode combinations.
\end{itemize}

Despite these limitations, the CWRU dataset remains valuable for benchmarking and comparative studies due to its widespread adoption, well-documented characteristics, and availability. Our results on this dataset provide meaningful insights while acknowledging the need for validation on additional datasets for comprehensive generalization assessment.
