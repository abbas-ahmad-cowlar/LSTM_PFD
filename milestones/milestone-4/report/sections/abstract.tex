Bearing failures are among the leading causes of unplanned downtime in rotating machinery, making early and accurate fault detection critical for industrial operations. This study presents a comprehensive investigation of deep learning approaches for automated bearing fault diagnosis using vibration signal analysis. We systematically evaluate three distinct methodologies: Convolutional Neural Networks (CNNs) for spatial pattern recognition, Long Short-Term Memory (LSTM) networks for temporal sequence modeling, and a novel configurable hybrid architecture that combines both approaches.

Using the Case Western Reserve University bearing dataset comprising 1,430 vibration signals across 11 fault categories, we implement and compare multiple architectural variants. Our CNN-based approach (Milestone 1) employs 15+ architectures including ResNet and EfficientNet families, achieving \todo{XX.XX\%} classification accuracy through effective spatial feature extraction. The LSTM-based approach (Milestone 2) utilizes both unidirectional and bidirectional recurrent networks, demonstrating \todo{XX.XX\%} accuracy by capturing temporal dependencies in sequential data. Most notably, our hybrid CNN-LSTM framework (Milestone 3) introduces a configurable architecture allowing arbitrary combinations of CNN backbones with LSTM types, achieving \todo{XX.XX\%} accuracy while providing flexibility for application-specific optimization.

Experimental results reveal that \todo{[comparative analysis of the three approaches will be inserted here after training completion]}. The hybrid architecture demonstrates particular effectiveness in \todo{[specific scenarios/fault types]}. We provide detailed performance metrics including per-class precision, recall, F1-scores, and confusion matrices for all approaches. Computational efficiency analysis shows \todo{[inference time comparisons]} with \todo{[model size comparisons]}.

This work contributes to the field of intelligent fault diagnosis by: (1) providing a systematic comparison of CNN, LSTM, and hybrid approaches on a standardized bearing dataset, (2) introducing a flexible hybrid architecture framework enabling rapid experimentation with different backbone combinations, (3) delivering production-ready implementations with comprehensive documentation, and (4) establishing performance benchmarks for future research. Our findings have direct implications for industrial predictive maintenance systems, enabling more reliable condition monitoring and reducing operational costs through early fault detection.
